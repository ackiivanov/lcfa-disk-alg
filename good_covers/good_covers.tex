% Run only with XeLaTeX!
%---------------------------------------------------------------------
%	FORMATING AND TYPESETTING
%---------------------------------------------------------------------

% Set document class
\documentclass[12pt,a4paper]{article}

% Option to change margin size
\usepackage[margin=2cm]{geometry}

% Font and encoding settings
\usepackage[T1]{fontenc}
%\usepackage[utf8]{inputenc} not necessary with xelatex?
\usepackage{yfonts} % germanic fonts (fraktur...)
%\defaultfontfeatures{Ligatures=TeX} %??

% Footnotes
\usepackage[stable]{footmisc}

% Indentation and spacing
\usepackage{indentfirst} % indentation
\usepackage{enumitem}[leftmargin=0pt] % spacing
\setlist{nosep}

% Multi column environment
\usepackage{multicol}

% Formating of titles
\usepackage{titlesec}
\titleformat{\section}
{\normalfont\large\bfseries}{\thesection}{1em}{}
\titleformat{\subsection}
{\normalfont\normalsize\bfseries}{\thesubsection}{1em}{}
\titleformat{\subsubsection}
{\normalfont\small\bfseries}{\thesubsubsection}{1em}{}
\titleformat{\paragraph}[runin]
{\normalfont\small\bfseries}{\theparagraph}{1em}{}
\titleformat{\subparagraph}[runin]
{\normalfont\small\bfseries}{\thesubparagraph}{1em}{}
\usepackage[titletoc]{appendix}

% Math typesetting packages
\usepackage{amsmath, amssymb, amsthm, mathtools, commath}
\usepackage[warnings-off={mathtools-colon,mathtools-overbracket}]{unicode-math} % makes \mathbb{1} work and changes font of mathbb
\usepackage[mathscr]{euscript} % euler script font
\usepackage{tikz-cd} % tikz commutative diagrams
\usepackage{thmtools, thm-restate} % theorem tools, especially restatable

\AtBeginDocument{\renewcommand\setminus{\smallsetminus}} % \setminus doesn't work in unicode-math 

%---------------------------------------------------------------------


%---------------------------------------------------------------------
%	GRAPHICS
%---------------------------------------------------------------------

\usepackage{float}
\usepackage{caption}
\usepackage{graphicx}
\graphicspath{{images/}}

\usepackage{pdfpages} % for inserting pdf pages


%---------------------------------------------------------------------


%---------------------------------------------------------------------
%	BIBLIOGRAPHY
%---------------------------------------------------------------------

\usepackage[backend=biber,style=alphabetic]{biblatex}
\renewcommand*{\bibfont}{\small}
\addbibresource{text.bib}

%---------------------------------------------------------------------


%---------------------------------------------------------------------
%	HYPERLINKS
%---------------------------------------------------------------------

% Define the colors
\usepackage{xcolor}
\definecolor{brown}{HTML}{CD853F}
\definecolor{blue}{HTML}{4169E1}
\definecolor{green}{HTML}{2E8B57}

% Setup options for hyperlinks
\usepackage{hyperref}
\hypersetup{
colorlinks=true, linkcolor=black, breaklinks=true,
linktoc=all, bookmarksnumbered=true,
urlcolor=brown, linkcolor=blue, citecolor=green, % Link colors
}

% Referencing package
\usepackage{cleveref}
\crefname{section}{\S\!}{\S\S\!}
\Crefname{section}{\S\!}{\S\S\!}


%---------------------------------------------------------------------
%	THEOREM STYLES
%---------------------------------------------------------------------

% Define a dummy counter to keep count for all sub-environments
\newcounter{counter} \numberwithin{counter}{section}

% Define theorem styles here based on the definition style
\theoremstyle{definition}
\newtheorem{definition}[counter]{Definition}
\newtheorem{construction}[counter]{Construction}
\newtheorem{observation}[counter]{Observation}

% Define theorem styles here based on the plain style
\theoremstyle{plain} 
\newtheorem{theorem}[counter]{Theorem}
\newtheorem*{theorem*}{Theorem}
\newtheorem{corollary}[counter]{Corollary}
\newtheorem{lemma}[counter]{Lemma}
\newtheorem{proposition}[counter]{Proposition}
\newtheorem*{proposition*}{Proposition}
\newtheorem{conjecture}[counter]{Conjecture}


% Define theorem styles here based on the remark style
\theoremstyle{remark}
\newtheorem{example}[counter]{Example}
\newtheorem{remark}[counter]{Remark}


%---------------------------------------------------------------------


%---------------------------------------------------------------------
%	NEW COMMANDS
%---------------------------------------------------------------------

% general target category
\newcommand{\catC}{\mathscr{C}}

% categories of factorization algebras
\newcommand{\falg}{\mathscr{F} \mathsf{Alg}}
\newcommand{\lcfa}{\mathscr{F} \mathsf{Alg}^{\mathsf{lc}}}

% category of manifolds
\newcommand{\mfld}[1][s]{%
    \ifx r#1\mathsf{Mfld}\else
    \ifx s#1\mathscr{M} \mathsf{fld}\else
    \mathrm{Illegal~option}%
    \fi\fi
}

% category of disks
\newcommand{\disk}[1][s]{%
    \ifx r#1\mathsf{Disk}\else
    \ifx s#1\mathscr{D} \mathsf{isk}\else
    \mathrm{Illegal~option}%
    \fi\fi
}

% category of opens
\newcommand{\opens}[1][s]{%
    \ifx r#1\mathsf{Opns}\else
    \ifx s#1\mathscr{O} \mathsf{pns}\else
    \mathrm{Illegal~option}%
    \fi\fi
}

% category of cosheaves
\newcommand{\csheaves}[1][n]{%
    \ifx n#1 \mathsf{c} \mathscr{S} \mathsf{hv}\else
    \mathsf{c} \mathscr{S} \mathsf{hv}^{\mathsf{#1}}
    \fi
}

% category of infinity categories
\newcommand{\cat}{\mathscr{C} \mathsf{at}_{\infty}}

% template for category of algebras
\newcommand{\alg}[1]{\mathscr{A} \mathsf{lg}_{#1}}

% template for category of locally constant algebras
\newcommand{\lcalg}[1]{\mathscr{A} \mathsf{lg}^{\mathsf{lc}}_{#1}}

% half-open interval
\newcommand{\hoint}{\mathbb{R}_{\geq 0}}

% Cech complex
\newcommand{\cech}{\check{\mathsf{C}}}

% Category of basics
\newcommand{\bsc}{\mathscr{B} \mathsf{sc}}
\newcommand{\bstr}{\mathscr{B}}

% Open cone
\newcommand{\cone}[1][n]{%
    \ifx b#1\overline{\mathsf{C}}\else
    \ifx n#1\mathsf{C}\else
    \mathrm{Illegal~option}%
    \fi\fi
}

\newcommand{\lrangle}[1]{\langle #1 \rangle}

%---------------------------------------------------------------------

\usepackage{subfiles}

\setcounter{section}{-1}

\begin{document}

% Set title, author and date
\title{\sc Good Covers of Stratified Spaces}
\author{Aleksandar Ivanov, Ödül Tetik}
\date{}
\maketitle

\tableofcontents

\section{Theorem}

\begin{theorem}
    Every stratified manifold has a good cover.
\end{theorem}

\subsection{Riemannian Metrics}

We do not claim that this is the most general structure that deserves the name `Riemannian metric' on a stratified space. This is why we opt to add the adjective `tame' to the name. Whatever that most general definition of a Riemannian metric is it should certainly include, as an example, the structure that we consider here. For the purposes of defining good covers this restricted structure is enough.

\begin{remark}
    Recall that, by ......., every stratum of a stratified manifold is a smooth manifold. Furthermore, by ......., every stratum has a tubular neighborhood in the stratified manifold that is characterized by the link.
\end{remark}


As a preliminary, we consider the construction of a Riemannian metric on a stratified manifold of depth 1. The general construction, that we will describe in ......, proceeds by induction on this case.


\begin{definition}
    Let $M$ be a stratified manifold of depth 1, so that $M_0$ is its lower stratum, $M_1$ is its higher one, and the link is $M_0 \twoheadleftarrow{\pi} L \xhookrightarrow{\iota} M_1$. A \emph{tame Riemannian metric} $g$ on $M$ is prescribed by Riemannian metrics $g_0$ and $g_1$ on the respective strata such that under the canonical isomorphism
    %
    \begin{equation}
        \mathsf{T}L \cong \iota^*\mathsf{T}M_1 \oplus \mathsf{T}M_0 \oplus \underbar{\mathbb{R}}
    \end{equation}
    %
    the pullback metric $\iota^* g_1$ on the link decomposes as
    %
    \begin{equation}
        (r^2 g_1, g_0, dr^2).
    \end{equation}
\end{definition}














\begin{definition}\label{def:Riem_met_on_strat}
    A tame Riemannian metric on a stratified manifold is ... 
\end{definition}


\begin{example}
    If a stratified manifold $M$ is such that it has an underlying smooth manifold $\bar{M}$, like in the case of stratified manifolds created through a filtration of closed subsets, then a Riemannian metric $g_{\bar{M}}$ on $\bar{M}$ provides the data of a tame Riemannian metric on the stratified manifold $M$, as in definition \Cref{def:Riem_met_on_strat}. Namely, .......
\end{example}




\begin{definition}
    Let $\gamma: [0,1] \xrightarrow{} M$ be a curve in the stratified manifold $M \xrightarrow{} P$, and let $M$ be equipped with a tame Riemannian metric. The length of $\gamma$ is defined to be
    %
    \begin{equation}
        \mathrm{len}_{g}(\gamma) := \sum_{p \in P} \mathrm{len}_{g_p} (\gamma|_{M_p}),
    \end{equation}
    %
    the sum of the lengths in each stratum. If the metric is clear from context we will drop the subscript notation.
\end{definition}


\begin{lemma}\label{lem:length_space}
    Let $(M,g)$ be a stratified manifold with a tame Riemannian metric. Then $M$ can be made into a length space, by equipping it with the intrinsic length metric $d_g$, so that the distance between two points is the infimum of the lengths of admissible curves joining them.
\end{lemma}


\begin{proposition}
    Let $(M, g)$ be a connected stratified manifold with a tame Riemannian metric. Then every two points of $M$ can be connected by a minimizing curve.
\end{proposition}


\begin{proof}
    By \Cref{lem:length_space}, $(M,d_g)$ is a length space, where $d_g$ is the intrinsic metric generated by the tame Riemannian metric $g$. By virtue of being a stratified manifold it is also a locally compact topological space. Then using the Hopf--Rinow--Cohn-Vossen theorem ....... for a locally compact length space, and its immediate corollaries, we know that what we need to show is that every closed and bounded subset of $M$ is compact.........
\end{proof}








\end{document}
