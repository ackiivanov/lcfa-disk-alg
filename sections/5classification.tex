\documentclass[../text]{subfiles} 

\begin{document}

\section{Towards Classifying \texorpdfstring{$\lcfa$}{FAlglc} on Defect Manifolds}\label{ch:classif_defect_mfld}

Our goal in this section is to lay out a possible roadmap for classifying locally constant factorization algebras on stratified manifolds. Specifically we will focus on the case of stratified manifolds $M_{\Sigma}$, which are described by a smooth, $n$-dimensional manifold $M$ together with a distinguished smooth, properly embedded, $d$-dimensional submanifold $\Sigma \xhookrightarrow{} M$, such that $d < n$. In other words we are considering objects $M_{\Sigma} \in \mfld(\mathsf{D}_{d \subset n})$ as discussed in \cref{def:mfld_disk_dn}. More general statements can be achieved by iterating the procedure and explicating the information of more defects, stratum by stratum, in the stratified manifold.

\begin{remark}\label{rem:disk_not_enough}
    Having already considered the classification of disk algebras with $\mathsf{D}_{d \subset n}^*$-structure in \cref{thm:classif_disk_d_under_n*_alg}, one might wonder whether the classification we are after is already done by simply using the factorization homology functor, and porting over the classification of disk algebras to the setting of locally constant factorization algebras. There are two issues with this direction of inquiry. One problem is that not all $\mathsf{D}_{d \subset n}$-manifolds are frameable since this is not even the case for smooth manifolds; framed disk algebras like in \cref{thm:classif_disk_d_under_n*_alg} can only evaluate framed stratified manifolds.
    
    The second problem is that having classified all $\disk(\bstr)$-algebras does not immediately give a classification of $\disk_{/M}$-algebras for a given manifold $M$. There is a functor
    %
    \begin{equation}
        \alg{\disk(\bstr)}(\catC) \xrightarrow{\quad} \alg{\disk(\bstr)_{/M}} (\catC),
    \end{equation}
    %
    which is induced by the forgetful functor $\disk_{/M} \rightarrow \disk(\bstr)$, but this functor will, in general, not be an equivalence. We can see this, for example, by using \cref{rem:disk_b=disk_bsc} which tells us that the left-hand side depends on $\bstr$, while the right-hand side doesn't. 
\end{remark}



\subsection{Euclidean Spaces with Defects}


In \cref{ssec:lcfa_on_Rn} we discussed locally constant factorization algebras on Euclidean spaces of different dimension $n$, and we saw that they presented the data of $\mathbb{E}_n$-algebras. In \cref{ssec:lcfa_on_ints}, on the other hand we focused on 1 dimensional manifolds, but with different possible stratification structures, and we saw that the data encoded in defects was a module structure. Here, in the spirit of this section, we work to extend both of these developments to the case of Euclidean spaces with the specific kind of defect that we have limited ourselves to, namely, $\mathbb{R}^{d \subset n}$.

Despite \cref{rem:disk_not_enough}, we know from \cref{lem:disk/M_to_disk} that sometimes the classification results for disk algebras can be used for factorization algebras. In particular, this is the case for $\mathbb{R}^{d \subset n}$, since it is final in the $\infty$-category of basics $\bstr = \mathsf{D}_{d \subset n}^*$, as long as $d < n - 1$. This gives us
%
\begin{equation}\label{eq:focus_on_disk_dn}
    \lcfa_{\mathbb{R}^{d \subset n}} (\catC) \simeq \alg{\disk (\mathsf{D}_{d \subset n}^*)} (\catC).
\end{equation}

In fact, technically, the only cases we need to consider are when $d=0$ --- the case of pointed Euclidean spaces, which are alternatively denoted $\mathbb{R}^n_*$. This is because of the isomorphism
%
\begin{equation}
    \mathbb{R}^{d \subset n} \cong \mathbb{R}^{n-d}_* \times \mathbb{R}^d,
\end{equation}
%
together with \cref{prop:exp_of_products_lc}, which describes locally constant factorization algebras on product spaces. Since $\mathbb{R}^n_*$ is a basic the collar-gluing property \cref{cor:gluing_lcfas} can't help us to analyze the situation. We are left with two cases: $n=1$ and $n > 1$. To get a better grasp on the data, we first explore the more complicated case of $n > 1$ by examining the $\infty$-operad $\disk(\mathsf{D}_{0 \subset n}^*)$, when $n > 1$. 

\begin{observation}\label{obs:explore_D*}
    For the purposes of the observation we will shorten notation to $\mathscr{D}_* := \disk (\mathsf{D}_{0 \subset n}^*)$. Since the defect is a point there can be at most one disk that contains it per morphism. Thus, the morphism spaces to consider are
    %
    \begin{align}
        &\mathsf{Hom}_{\mathscr{D}_*} ((\mathbb{R}^n)^{\sqcup i}, \mathbb{R}^n),& &\mathsf{Hom}_{\mathscr{D}_*} ((\mathbb{R}^n)^{\sqcup i}, \mathbb{R}^n_*)& &\mathrm{and}& &\mathsf{Hom}_{\mathscr{D}_*} ((\mathbb{R}^n)^{\sqcup i} \sqcup \mathbb{R}^n_*, \mathbb{R}^n_*).&
    \end{align}
    %
    The first of these is equivalent to the space $\mathsf{Emb}^*((\mathbb{R}^n)^{\sqcup i}, \mathbb{R}^n)$ of framed open embeddings, which makes it is obvious that restricting to non-defect disks would give the data of an $\mathbb{E}_n$-algebra. This is related to \cref{lem:ff_functor_to_refinement}, which says that forgetting the stratification results in obtaining a full $\infty$-subcategory of locally constant factorization algebras. When $i=0$, the third space above is equivalent to a point, which makes it clear that restricting exclusively to pointed disks $\mathbb{R}^n_*$ would give rise to plain objects\footnote{We've omitted mentioning the fact that, because of morphisms like $\emptyset \xrightarrow{} U$, all of the above algebraic objects will be pointed. We have done this to avoid confusion with the use of pointed in the sense of a point defect of the space.} (or $\mathbb{E}_0$-algebras). The structure that connects these two algebras is encoded in the rest of the morphism spaces. In fact the second and third space above are equivalent,
    %
    \begin{equation}
        \mathsf{Hom}_{\mathscr{D}_*} ((\mathbb{R}^n)^{\sqcup i}, \mathbb{R}^n_*) \simeq \mathsf{Hom}_{\mathscr{D}_*} ((\mathbb{R}^n)^{\sqcup i} \sqcup \mathbb{R}^n_*, \mathbb{R}^n_*) \simeq \mathsf{Emb}^* ((\mathbb{R}^n)^{\sqcup i}, \mathbb{R}^n \setminus \{0\}),
    \end{equation}
    %
    for all $i \in \mathbb{N}$, where $\mathsf{Emb}^* ((\mathbb{R}^n)^{\sqcup i}, \mathbb{R}^n \setminus \{0\})$ is the space of framed embeddings of $i$ $n$-dimensional disks into the punctured $\mathbb{R}^n$. All of these facts are partially an elaboration of the fact that $\mathsf{D}_{0 \subset n}^*$ is equivalent to $\Delta^1 = \{0 \xrightarrow{} 1\}$ as $\infty$-categories (saying nothing about the right fibration to $\bsc$). 

    The isomorphism $\mathbb{R}^n \setminus \{ 0 \} \cong S^{n-1} \times \mathbb{R}$ together with an examination of \cite[prop.3.16]{francis2013}, which we recalled in \cref{univ_env_for_En}, and its proof leads us to suggest that the $\mathbb{E}_n$-algebra $A$ and the $\mathbb{E}_0$-algebra $B$ are further such that $B$ is a module over the universal enveloping algebra $U_A \simeq \int_{S^{n-1}} A$ as an $\mathbb{E}_1$-algebra. Remembering that we limited the whole discussion to the case when $n>1$, we propose:
\end{observation}

\begin{theorem}\label{thm:classif_Rn*}
    Let $n$ be bigger than 1. There is an equivalence between the $\infty$-category of locally constant factorization algebras on $\mathbb{R}^n_*$ and the $\infty$-category of $\mathbb{E}_n$-modules that fits into the commutative diagram
    %
    \begin{equation}
        \begin{tikzcd}
            \mathsf{Mod}^{\mathbb{E}_n} (\catC) \arrow[r, "\simeq"] \arrow[d] & \lcfa_{\mathbb{R}^n_*} (\catC) \arrow[d, "(\ref{lem:ff_functor_to_refinement})"] \\
            \alg{\mathbb{E}_n} (\catC) \arrow[r, "\simeq"] & \lcfa_{\mathbb{R}^n} (\catC).
        \end{tikzcd}
    \end{equation}
\end{theorem}

\begin{remark}
    In the above $\mathsf{Mod}^{\mathbb{E}_n}$ is the $\infty$-category of $\mathbb{E}_n$-modules, which is fibered over $\alg{\mathbb{E}_n}$ the $\infty$-category of $\mathbb{E}_n$-algebras. Informally speaking, the objects of $\mathsf{Mod}^{\mathbb{E}_n}$ are pairs $(A, M)$ consisting of an $\mathbb{E}_n$-algebra $A$ and an $A$-module $M$. However, to make things precise we, as always, use the definitions in \cite[ch.3]{lurie_ha}.
\end{remark}


\begin{proof}
    From \cref{eq:focus_on_disk_dn} we know that when $n > 1$ we have
    %
    \begin{equation}
        \lcfa_{\mathbb{R}^n_*} (\catC) \simeq \alg{\mathscr{D}_*} (\catC),
    \end{equation}
    %
    where we used the notation from \cref{obs:explore_D*}. Specializing \cref{thm:classif_disk_d_under_n*_alg} to the case at hand, and rewriting gives the left pullback square of the diagram
    %
    \begin{equation}
        \begin{tikzcd}
            \mathsf{Mod}^{\mathbb{E}_n}_A (\catC) \arrow[dr, phantom, "\ulcorner", very near start] \arrow[r] \arrow[d] & \alg{\mathscr{D}_*} (\catC) \arrow[d] \arrow[r, "\simeq"] & \lcfa_{\mathbb{R}^n_*} (\catC) \arrow[d, "(\ref{lem:ff_functor_to_refinement})"] \\
            * \arrow[r, "\{A\}"] & \alg{\mathbb{E}_n} (\catC) \arrow[r, "\simeq"] & \lcfa_{\mathbb{R}^n} (\catC).
        \end{tikzcd}
    \end{equation}
    %
    The right square obviously commutes by construction. Since, by definition, $\mathsf{Mod}^{\mathbb{E}_n} (\catC)$ is fibered over $\alg{\mathbb{E}_n} (\catC)$, with fibers $\mathsf{Mod}^{\mathbb{E}_n}_A (\catC)$ for $A$ an $\mathbb{E}_n$-algebra, the above diagram gives exactly the desired statement.
\end{proof}

\begin{remark}
    The study of locally constant factorization algebras on $\mathbb{R}^n_*$ was also undertaken in \cite[sec.6.3]{ginot2015}. \Cref{thm:classif_Rn*}, as it appears above, is slightly different in its statement compared to the corresponding \cite[cor.8]{ginot2015}. We believe that this is due to the fact that using the homotopy pullback is not what the proof provided in the reference actually does. The remaining discrepancy is then explained by our \cref{cor:gluing_lcfas}, which glues locally constant factorization algebras.
\end{remark}

\begin{remark}
    The case $n=1$ is special because the higher stratum $\mathbb{R} \setminus \{0\}$ is now disconnected. Looking back at the construction leading up to \cref{prop:hoint_gives_modules} and at \cref{rem:left_or_right_module} we can guess that a statement like the following can hold true:
\end{remark}

\begin{proposition}
    There is an equivalence between the $\infty$-category of locally constant factorization algebras on $\mathbb{R}_*$ and the $\infty$-category consisting of two $\mathbb{E}_1$-algebras $A$ and $B$ together with a pointed, left $A \otimes B^\mathsf{op}$-module $M$
    %
    \begin{equation}
        \lcfa_{\mathbb{R}_*} \simeq \mathsf{LMod} (\catC) \bigtimes_{\catC_{\mathbb{1}/}} \mathsf{RMod} (\catC),
    \end{equation}
    %
    where the functors $\mathsf{LMod}(\catC) \rightarrow \catC_{\mathbb{1}/}$ and $\mathsf{RMod}(\catC) \rightarrow \catC_{\mathbb{1}/}$ are the ones giving the underlying (pointed) object of a module.
\end{proposition}



\subsection{The General Case of \texorpdfstring{$M_{\Sigma}$}{M Sigma}}

Locally, near the defect, the locally constant factorization algebra is, of course, going to look like a locally constant factorization algebra over $\mathbb{R}^{d \subset n}$. However, it is possible that the global topology introduces some twisting of this local understanding. Due to \cref{cor:gluing_lcfas} and the properties of stratified manifolds, in any particular case, it's possible to find some list of collar-gluings that reduce the problem down to the local case, but this doesn't tell us anything about the general structure. Despite this, there is a particular collar-gluing that is the obvious first step if we're looking at a general manifold $M_\Sigma$. We will now describe this.

The data of an embedding $\Sigma \xhookrightarrow{} M_\Sigma$ allows us to construct a regular neighborhood of our defect, echoing the tubular neighborhood theorem, now in the case of stratified spaces.
%
\begin{theorem}[{\cite[prop.8.2.3, prop.8.2.5]{aft_localstrut}}]
    Let $\Sigma \xhookrightarrow{} M_\Sigma$ be a proper, constructible embedding of stratified manifolds. There exists a stratified map
    %
    \begin{equation}
        \mathsf{C}(\pi) \rightarrow M_\Sigma
    \end{equation}
    %
    under $\Sigma \xhookrightarrow{} M_\Sigma$ such that:
    %
    \begin{enumerate}
        %\item the map is a refinement onto its image,
        \item the image is open, and
        \item $\mathsf{C}(\pi)$ is the fiberwise open cone of the constructible bundle $\mathsf{L}_{\Sigma} \xrightarrow{\pi} \Sigma$ giving the link.
    \end{enumerate}
    %
    If $\Sigma$ is a stratum of $M_\Sigma$ then the map $\mathsf{C}(\pi) \rightarrow M_\Sigma$ is even an open embedding.
\end{theorem}

\begin{remark}
    In more detail, specialized to our situation, the regular neighborhood $\mathsf{C}(\pi)$ is constructed as the pushout of stratified manifolds
    %
    \begin{equation}
        \begin{tikzcd}
            \mathsf{L}_{\Sigma} \times \{0\} \arrow[r, "\pi"] \arrow[d, "\mathrm{id} \times 0"', hook] & \Sigma \arrow[d] \\
            \mathsf{L}_{\Sigma} \times \hoint \arrow[r] & \mathsf{C}(\pi) \arrow[ul, phantom, "\lrcorner", very near start],
        \end{tikzcd}
    \end{equation}
    %
    where $\mathsf{L}_{\Sigma}$ is the link of $\Sigma$ in $M_{\Sigma}$ (see \cite{aft_localstrut}). In our case of interest where the stratified manifold is $M_{\Sigma}$, the link $\mathsf{L}_{\Sigma}$ takes a simple form; it can always be constructed as the sphere bundle of the normal bundle of $\Sigma$ when embedded into $M$ as smooth manifolds.
\end{remark}

\begin{example}
    For the case of $(\Sigma \xhookrightarrow{} M) = {S^1 \xhookrightarrow{} \mathbb{R}^3}$, i.e. a circle defect in Euclidean 3-space, the link $\mathsf{L}_{\Sigma}$ would be a torus.
\end{example}

\begin{construction}
    The existence of $\mathsf{C}(\pi)$ allows us to define a collar-gluing for the stratified manifold $M_\Sigma$
    %
    \begin{equation}\label{eq:M_Sigma_decomposition}
        M_\Sigma \cong \left( M \setminus \Sigma \right) \coprod_{\mathsf{L}_\Sigma \times \mathbb{R}} \mathsf{C}(\pi),
    \end{equation}
    %
    where the map $\mathsf{L}_\Sigma \times \mathbb{R} \rightarrow M \setminus \Sigma$ is provided by $\mathsf{L}_\Sigma \times \mathbb{R} \cong \mathsf{L}_\Sigma \times \mathbb{R}_{\gneq 0} \cong \mathsf{C}(\pi) \setminus \Sigma \xhookrightarrow{} M \setminus \Sigma$. We also note that the properness of the embedding $\Sigma \xhookrightarrow{} M$ makes $M \setminus \Sigma$ an open submanifold of $M$.
\end{construction}


\Cref{cor:gluing_lcfas} allows us to use the collar-gluing from above to write the equivalence of $\infty$-categories
%
\begin{equation}
    \lcfa_{M_\Sigma} (\catC) \simeq \lcfa_{M \setminus \Sigma} (\catC) \bigtimes_{\lcfa_{\mathsf{L}_\Sigma \times \mathbb{R}} (\catC)} \lcfa_{\mathsf{C}(\pi)} (\catC),
\end{equation}
%
as a first step in the classification of locally constant factorization algebras on $M_{\Sigma}$. Since we have focused on stratified manifolds with the particular form $M_{\Sigma}$, the locally constant factorization algebras $\lcfa_{M \setminus \Sigma}$ in this decomposition already live on a smooth manifold, rather than on a stratified one. Similarly, this is the case for the locally constant factorization algebras $\lcfa_{\mathsf{L}_\Sigma \times \mathbb{R}}$. The more prescient question is about the classification of locally constant factorization algebras on the one remaining stratified space $\mathsf{C}(\pi)$. 

By \cite[ex.3.6.6]{aft_localstrut} $\mathsf{C}(\pi)$ is always a bundle over $\Sigma$, fibered over basics. For our particular setup this means that $\mathsf{C}(\pi)$ a fiber bundle over $\Sigma$ with typical fiber $\mathbb{R}^{n-d}_*$. Thus, the classification comes down to the classification of locally constant factorization algebras on stratified fiber bundles.

It is outside the scope of this work to finish the classification as outlined above. However, there is a special case that we can already tackle, which is the case of framed stratified manifolds. This is because, as discussed in \cref{def:framed_d_under_n_structure}, $\mathsf{D}_{d \subset n}^*$-manifolds come with a trivialization of the normal bundle of the distinguished submanifold $\Sigma$. This, in turn implies trivializations for $\mathsf{C}(\pi)$ and $\mathsf{L}_{\Sigma} \times \mathbb{R}$ as bundles over $\Sigma$ with typical fibers $\mathbb{R}^{n-d}_*$ and $S^{n-d-1} \times \mathbb{R}$, respectively. In other words, there are isomorphisms
%
\begin{align}
    &\mathsf{C}(\pi) \cong \Sigma \times \mathbb{R}^{n-d}_*& &\mathrm{and}& &\mathsf{L}_{\Sigma} \times \mathbb{R} \cong \Sigma \times S^{n-d-1} \times \mathbb{R}.&
\end{align}

However, \cref{prop:exp_of_products_lc} already gives us a way to classify locally constant factorization algebras on product spaces, if we know them on the product components. Having classified locally constant factorization algebras on the pointed Euclidean space in \cref{thm:classif_Rn*}, we already have the result that
%
\begin{equation}
    \lcfa_{\mathsf{C}(\pi)} (\catC) \simeq \mathsf{Mod}^{\mathbb{E}_{n-d}} (\lcfa_{\Sigma} (\catC)).
\end{equation}
%
Similarly, we also know that for the link we have
%
%
\begin{equation}
    \lcfa_{\mathsf{L}_{\Sigma} \times \mathbb{R}} (\catC) \simeq \lcfa_{\mathbb{R}^{n-d} \setminus \{ 0\} } (\lcfa_{\Sigma} (\catC)),
\end{equation}
%
exactly in such a way that the functor $\mathsf{Mod}^{\mathbb{E}_{n-d}} (\lcfa_{\Sigma} (\catC)) \xrightarrow{} \lcfa_{\mathbb{R}^{n-d} \setminus \{ 0\} } (\lcfa_{\Sigma} (\catC))$ is the one given by $\int_{S^{n-d-1}} -$, which to an algebra assigns its universal enveloping algebra.

Putting it all together, we have the equivalence
%
\begin{equation}
    \lcfa_{M_\Sigma} (\catC) \simeq \lcfa_{M \setminus \Sigma} (\catC) \bigtimes_{\lcfa_{\mathbb{R}^{n-d} \setminus \{0\}} (\lcfa_{\Sigma} (\catC))} \mathsf{Mod}^{\mathbb{E}_{n-d}} (\lcfa_{\Sigma} (\catC)).
\end{equation}
%
where the only functor that we haven't mentioned above is $\lcfa_{M \setminus \Sigma} (\catC) \xrightarrow{} \lcfa_{\mathbb{R}^{n-d} \setminus \{ 0\} } (\lcfa_{\Sigma} (\catC))$, which is given by restriction to the fattened link.

This description confirms the observation that what a locally constant factorization algebra on a defect manifold encodes is an algebra away from the defect, an algebra on the defect and a module structure relating them. Additionally, it exactly specifies what that module structure should be given by. In the case where the stratified manifold isn't framed we expect that the above structure will be `twisted' as seen in \cref{ex:cylinder_mobius_band} for the case of the M\"obius band.

\end{document}