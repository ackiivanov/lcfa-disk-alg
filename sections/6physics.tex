\documentclass[../text]{subfiles} 

\begin{document}



\section{Applications to Physical Examples}\label{ch:physics}

As we mentioned previously, a lot of the work on factorization algebras can be given a physics interpretation. This is especially visible in \cite{cg2016}, where the motivating examples come from quantum mechanics and quantum field theory, and specifically from the deformation quantization approach. We will introduce some of these examples here, following the exposition in \cite{cg2016}.


\begin{example}[A Barebones Model of Quantum Mechanical Observables]\label{ex:barebones_qm}
    We begin by considering quantum mechanics. Our goal will be to show that the observables of quantum mechanical systems naturally organize to form a prefactorization algebra.

    To avoid issues around measurement in quantum mechanics, we will say that the observables $\mathsf{Obs}(U)$ of a quantum system that occur in a given time period $U \subset \mathbb{R}$ are all the possible ways in which an arbitrary test system can be changed by being coupled to our system. Since we are not being precise the exact definition of what observables are will not matter a lot for now; it is rather their structure and operations that we care about. There are two natural requirements we make on the observable. For one, we require them to be defined for every time interval $U \subset \mathbb{R}$. Secondly, due to the superposition principle, for example, the minimal requirement of linearity is that $\mathsf{Obs}(U)$ is a complex vector space for each $U$.

    We notice that with this setup, if $V \subset \mathbb{R}$ is a smaller time interval contained in $U \subset \mathbb{R}$ then there is a natural restriction map
    %
    \begin{equation}
        \mathsf{Obs} (U) \rightarrow \mathsf{Obs} (V).
    \end{equation}
    %
    In effect, we have defined a copresheaf $\mathsf{Obs}: \opens[r](\mathbb{R}) \rightarrow \mathsf{Vect}_{\mathbb{C}}$ of complex vector spaces on $\mathbb{R}$.

    In the quantum setting, a key observation is that we cannot observe a system in two different ways at once, since observing changes the state of the system itself. However, we can do observations as long as they occur at different times. This is encapsulated as giving maps
    %
    \begin{equation}
        \mathsf{Obs} (U_1) \otimes \mathsf{Obs} (U_2) \xrightarrow{\quad} \mathsf{Obs} (U),
    \end{equation}
    %
    whenever $U_1$ and $U_2$ are disjoint and their union is contained in $U$. We also notice that for such a formulation we have immediately upgraded out target category $\mathsf{Vect}_{\mathbb{C}}$ to a symmetric monoidal category $\mathsf{Vect}_{\mathbb{C}}^{\otimes}$ equipped with the tensor product. Furthermore, we can do any finite number $n$ of measurements which would give us maps
    %
    \begin{equation}
        \bigotimes_{i=1}^{n} \mathsf{Obs} (U_i) \xrightarrow{\quad} \mathsf{Obs} (U),
    \end{equation}
    %
    under the same disjointness and containment criteria, but this maps also have to be compatible in a sense which essentially reproduces associativity. This all shows that under some very natural model assumptions the observables of a quantum mechanical system are organized into a $\mathsf{Vect}_{\mathbb{C}}$-valued prefactorization algebras on $\mathbb{R}$.

    In fact, we can expect even more to be true. Any observation that we get during a time interval $U$ can be refined so that we get the same observation during a shorter interval. We can think of this as increasing the coupling. In such a case the observables would in fact get the additional structure of being a locally constant, $\mathsf{Vect}_{\mathbb{C}}$-valued prefactorization algebra on $\mathbb{R}$.

    Another fact that this teaches us is that associative algebras only appear because of the topology of $\mathbb{R}$, since we know that had the dimensionality been higher like for $\mathbb{R}^n$ we would have gotten $\mathbb{E}_n$-algebras, and even more complicated algebras for other underlying manifolds.
\end{example}


\begin{example}[First Steps Towards Field Theories]
    The machinery of \cref{ex:barebones_qm} is not very advanced, and we can easily see that even if we switch to quantum field theory the previous considerations hold true. However, the way the details are fleshed out will definitely be different. The simplest example of this is the choice of target ($\infty$-)category. The target category that we chose in \cref{ex:barebones_qm} was the (ordinary) category of complex vector spaces. However, we immediately run into problems if we want to extend this to higher dimensions.

    One problem that arises is that $\mathsf{Vect}_{\mathbb{C}}$ is an ordinary category. This means that when viewed as an $\infty$-category its higher homotopies are all given in a trivial manner\footnote{The homotopies are exactly as specified by the nerve functor that turns an ordinary category into an $\infty$-category.}. Mathematically, this gives rise to the fact that the $\infty$-cateogries
    %
    \begin{equation}
        \alg{\mathbb{E}_n} (\mathsf{Vect}_{\mathbb{C}})
    \end{equation}
    %
    for $n \geq 2$ are all equivalent and all give rise to the concept of unital, commutative (and associative) complex algebras. $\mathsf{Vect}_{\mathbb{C}}$ simply does not have enough higher structure to differentiate between them. 

    In a similar vein, a physical observation that arises from gauge theories is that the vector spaces we work in are not plain but actually have the structure of cochain complexes. Namely, they are such that in a cochain complex of observables $V$, the `physical' observables are given by the zeroth cohomology $\mathsf{H}^0(V)$. This is because if $v \in V^0$ is an observable of degree 0, the equation $\mathrm{d} v = 0$ means that $v$ is compatible with the gauge symmetries of the theory, thus it has to be in the kernel of the differential. If we have two observables $v, v' \in V^0$, which are both in the kernel of the differential then they could be physically equivalent they always return the same result, which we interpret as $v' = v + \mathrm{d}u$ for some $u \in V^{-1}$, since the exact element $\mathrm{d} u$ can never physically be observed.

    Both of these observations point us to the $\infty$-category of chain (or cochain, depending on conventions) complexes $\mathscr{C}\mathsf{h}_{\mathbb{C}}$. Furthermore, the symmetric monoidal structure is the tensor product $\otimes$ since this is how independent quantum systems are combined.

    $\mathscr{C}\mathsf{h}_{\mathbb{C}}$ (in the cochain convention) has as objects cochain complexes and the morphism spaces can be given the following rough, combinatorial description that can be found in \cite{tanaka20}:
    %
    \begin{enumerate}
        \item The vertices of $\mathsf{Hom}(V, W)$ are maps of chain complexes $f: V \rightarrow W$, i.e. such that $\mathrm{d} f = f \mathrm{d}$.
        \item The edges of $\mathsf{Hom}(V, W)$ are given by triples $(f_0, f_1, H)$, where $f_0$ and $f_1$ are chain maps and $H$ is a chain homotopy, i.e. a degree -1 map $H: V \rightarrow W$ such that $f_1 - f_0 = \mathrm{d} H + H \mathrm{d}$.
        \item Simplices of $\mathsf{Hom}(V, W)$ of dimension $k$ are given by maps of degree $-k$, which exhibit higher homotopies. For example for $k=2$ we have the data
        %
        \begin{equation}
            \begin{tikzcd}[row sep = large, column sep = large]
                & f_1 \arrow[rd, "H_{12}"] \arrow[d, phantom, "G"] & \\
                f_0 \arrow[ru, "H_{01}"] \arrow[rr, "H_{02}"'] & {} & f_2,
            \end{tikzcd}
        \end{equation}
        %
        where $G$ is a degree -2 map that exhibits a homotopy between $H_{02}$ and $H_{12} + H_{01}$.
    \end{enumerate}
\end{example}


\begin{example}[Correlation Functions][\cite[sec.1.4.4]{cg2016}]
    There are some cases where we can very naturally extract correlation functions from the data of a factorization algebra. Working perturbatively, suppose we have a factorization algebra on a manifold $M$ valued in cochain complexes over the ring $\mathbb{R}[\hbar]$
    %
    \begin{equation}
        F \in \falg_M (\mathscr{C}\mathsf{h}_{\mathbb{R}[\hbar]}),
    \end{equation}
    %
    and suppose that
    %
    \begin{equation}\label{eq:H0_is_the_ring}
        \mathsf{H}^0 (F(M)) = \mathbb{R}[\hbar].
    \end{equation}
    %
    As noted in \cite[sec.1.4.4]{cg2016} this condition holds in some natural examples, like Chern--Simons theory on $\mathbb{R}^3$ or for massive scalar field theories on compact manifolds. In such a case, given disjoint open subsets $\{U_i \subset M\}_{i=1, \dots, n}$ the factorization algebra maps gives rise to a multilinear map
    %
    \begin{equation}
        \langle - \rangle: \bigtimes_{i=1}^{n} H^0(F(U_i)) \xrightarrow{\quad} \mathsf{H}^0 (F(M)) = \mathbb{R}[\hbar],
    \end{equation}
    %
    so that given physical observables $O_i \in \mathsf{H}^0(F(U_i))$, we get a formal power series $\langle O_1 \dots O_n \rangle \in \mathbb{R}[\hbar]$.

    If the condition in \cref{eq:H0_is_the_ring} is not satisfied, then there is a variant that works well for $\mathbb{R}^n$. In these cases we can define an $\mathbb{R}[\hbar]$-linear map
    %
    \begin{equation}
        \mathsf{H}^0(F(\mathbb{R}^n)) \rightarrow \mathbb{R}[\hbar]
    \end{equation}
    %
    called the vacuum, that is translation invariant and satisfies a type of cluster decomposition. We then proceed as above. The details of this can be found in \cite{cg2016}.
\end{example}

There are many more things that can be said about the connection of factorization algebras to physics, however the prerequisite material would be too great to state them properly. The two volumes of the book \cite{cg2016} go over the prerequisites and provide exactly this connection in various different ways.


\end{document}