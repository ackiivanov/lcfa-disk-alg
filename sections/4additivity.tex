\documentclass[../text.tex]{subfiles} 

\begin{document}


\section{Additivity of \texorpdfstring{$\lcfa$}{lcfa}s on Stratified Spaces}


In the special case of product spaces $X \times Y$, from the previous discussion we now know that the projections $\pi_1: X \times Y \rightarrow X$ and $\pi_2: X \times Y \rightarrow Y$, give us pushforwards. This is true even in the locally constant case since the projections are (even trivial) fiber bundles, which is a concept that is a special case of constructible bundles. For $\pi_1$, for example, this essentially uses the fact that if we can evaluate all open sets of $X \times Y$ then we can definitely evaluate the open sets that look like $U \times Y$, with $U \subset X$ an open set. But, in fact, we can do even more, because we can even evaluate the more granular open subsets $U \times V$ with $U \subset X$ and $V \subset Y$ opens. To capture this idea in formal terms we claim:
%
\begin{proposition}\label{prop:exp_of_products}
    Let $X$ and $Y$ be topological spaces. There is a functor from the $\infty$-category of prefactorization algebras on the product $X \times Y$ to the $\infty$-category of prefactorization algebras on $X$ valued in the $\infty$-category of prefactorization algebras on $Y$
    %
    \begin{equation}
        \bar{\pi}: \alg{\opens(X \times Y)}(\catC) \xrightarrow{\qquad} \alg{\opens(X)} (\alg{\opens(Y)} (\catC)).
    \end{equation}
    %
    This functor descends to factorization algebras too
    %
    \begin{equation}
        \bar{\pi}: \falg_{X \times Y}(\catC) \xrightarrow{\qquad} \falg_{X}(\falg_{Y}(\catC)).
    \end{equation}
\end{proposition}

\begin{remark}
    The conditions we imposed on $\catC$ allowed us, in \Cref{rem:sym_mon_inheritance}, to inherit a symmetric monoidal structure on the $\infty$-categories of factorization algebras of all varieties. It also allowed us to inherit the existence of colimits. This is the reason why the above can even be stated.
\end{remark}

\begin{proof}
    Indeed, given opens $U \subset X$ and $V \subset Y$, the values we assign are $(\bar{\pi}F) (U)(V) = F(U \times V)$. Fixing an open subset $U \subset X$, we obviously get a prefactorization algebra $(\pi_2|_U)_* (F|_U) \in \alg{\opens(Y)}$ by pushing forward with $\pi_2|_U: U \times Y \rightarrow Y$, which shows that $\bar{\pi}F$ is valued in $\alg{\opens(Y)}$. The only thing left to show is that $\bar{\pi} F$ has the structure of a prefactorization algebra on $X$. To do this, for each multi-morphism $(\{ U_i \}_{i \in I} \rightarrow U) \in \opens(X)$ we assign a map $\bar{\pi}F (\{ U_i \}_{i \in I}, U)$ of factorization algebras on $Y$. Such an algebra map is specified by giving its value on all opens of $Y$ separately. We can thus assign
    %
    \begin{equation}
        \bar{\pi}F (\{ U_i \}_{i \in I}, U)(V) = F(\{ U_i \times V \}_{i \in I}, U \times V),
    \end{equation}
    %
    which uses the projections $\pi_1|_V : X \times V \rightarrow X$ to assign $\pi_1|_V^{-1}(U) = U \times V$.

    It is not hard to check that the above also holds for the example of factorization algebras on topological spaces, by using \cite[\S7.2]{cg2016}. However, we'll focus on manifolds, in which case, the construction from above immediately works since we now have to do it for $\disk[r]_{/X}$ instead of $\mfld[r]_{/X}$. The only things to mention is that products of disks are again disks \cite[cor.3.4.9]{aft_localstrut}.
\end{proof}


\begin{theorem}[Additivity]\label{thm:additivity_lc}
    Let $M$ and $N$ be stratified manifolds. There is an equivalence of $\infty$-operads
    %
    \begin{equation}
        \disk_{/M} \otimes \disk_{/N} \xrightarrow{\ \ \simeq \ \ } \disk_{/M \times N},
    \end{equation}
    %
    between the tensor product of $\infty$-operads $\disk_{/M}$ and $\disk_{/N}$ and the $\infty$-operad $\disk_{/M \times N}$. In particular, there is an equivalence of $\infty$-categories of algebras
    %
    \begin{equation}
        \bar{\pi}: \lcfa_{M \times N}(\catC) \xrightarrow{\ \ \simeq \ \ } \lcfa_{M}(\lcfa_{N}(\catC)),
    \end{equation}
    which can be seen to be induced by the functor $\bar{\pi}$.
\end{theorem}

\begin{remark}
    A version of the above theorem also appears as \cite[prop.18]{ginot2015}. The equivalence proven there is for the case of smooth manifolds and relies on the local proof of \cite{lurie_ha} for $\mathbb{R}^n$. The statement at the level of $\infty$-operads is slightly stronger and, using \Cref{thm:disk_alg=lcfa}, it implies the statement for factorization algebras. This $\infty$-operadic statement, again for smooth manifolds, is found in \cite[ex.5.4.5.5]{lurie_ha}.
\end{remark}


\begin{proof}
    Following \cite[\S.2.2.5]{lurie_ha} the tensor product of $\infty$-operads is defined through a universal property. Namely, we define a bifunctor of $\infty$-operads to be a map $f: \mathscr{O}_1 \times \mathscr{O}_2 \xrightarrow{} \mathscr{O}$ such that:
    %
    \begin{enumerate}
        \item The diagram
        %
        \begin{equation}
            \begin{tikzcd}
                \mathscr{O}_1 \times \mathscr{O}_2 \arrow[r, "f"] \arrow[d] & \mathscr{O} \arrow[d] \\
                \mathsf{Fin}_* \times \mathsf{Fin}_* \arrow[r, "\wedge"] & \mathsf{Fin}_*
            \end{tikzcd}
        \end{equation}
        %
        commutes, and
        \item $f$ sends pairs of inert morphisms in $\mathscr{O}_1$ and $\mathscr{O}_2$ to inert morphisms in $\mathscr{O}$.
    \end{enumerate}
    %
    We denote the full $\infty$-subcategory of $\mathsf{Fun}_{\mathsf{Fin}_*}(\mathscr{O}_1 \times \mathscr{O}_2, \mathscr{O})$ consisting of the bifunctors of $\infty$-operads by $\mathsf{BiFunc}(\mathscr{O}_1, \mathscr{O}_2; \mathscr{O})$. The universal property of the tensor product is then that it is the $\infty$-operad $\mathscr{O}_1 \otimes \mathscr{O}_2$ together with a bifunctor of $\infty$-operads $f: \mathscr{O}_1 \times \mathscr{O}_2 \xrightarrow{} \mathscr{O}_1 \otimes \mathscr{O}_2$ such that composition with $f$ determines an equivalence
    %
    \begin{equation}
        \alg{\mathscr{O}_1 \otimes \mathscr{O}_2} (\mathscr{O}) \xrightarrow{\ \ \simeq \ \ } \mathsf{BiFunc}(\mathscr{O}_1, \mathscr{O}_2; \mathscr{O}),
    \end{equation}
    %
    for every $\infty$-operad $\mathscr{O}$\footnote{More details about the construction, its existence, and the fact that it captures a nontrivial concept can be found in the provided reference.}.

    For us this means that we can show the desired equivalence of $\infty$-operads if we provide a bifunctor of $\infty$-operads $\disk_{/M} \times \disk_{/N} \xrightarrow{} \disk_{/M \times N}$ that is an equivalence. Our candidate bifunctor for this purpose will be $\times$, the product of stratified spaces. We now explain what this means on a technical level.

    By \Cref{rem:details_disk/M} the definition of $\disk_{/M}$ as an $\infty$-operad is given by the pullback
    %
    \begin{equation}
        \begin{tikzcd}
            \disk_{/M} \arrow[rd, phantom, "\ulcorner", very near start] \arrow[r] \arrow[d] & (\disk_{/M})^{\amalg} \arrow[d] \\
            \disk(\bsc) \arrow[r] & \disk(\bsc)^{\amalg}
        \end{tikzcd},
    \end{equation}
    %
    where on the right-hand side $\disk_{/M}$ and $\disk(\bsc)$ are appearing in their role as $\infty$-categories through \cite[Con.2.4.3.1]{lurie_ha}, while on the left-hand side they're appearing in their role as $\infty$-operads. We note here that the construction of $\mathscr{O}^{\amalg}$ for any $\infty$-category $\mathscr{O}$ is given through a universal property from which it immediately follows that $\mathscr{O}_1^{\amalg} \times \mathscr{O}_2^{\amalg} \simeq (\mathscr{O}_1 \times \mathscr{O}_2)^{\amalg}$. Keeping this in mind together with the fact that products and pullbacks commute we can draw the following commutative diagram
    %
    \begin{equation}
        \begin{tikzcd}[column sep = 25]
            \disk_{/M} \times \disk_{/N} \arrow[rd, phantom, "\ulcorner", very near start] \arrow[r] \arrow[d] & (\disk_{/M})^{\amalg} \times (\disk_{/N})^{\amalg} \arrow[r, "\simeq"] \arrow[d] & (\disk_{/M} \times \disk_{/N})^{\amalg} \arrow[d] \\
            \disk(\bsc) \times \disk(\bsc) \arrow[r] & \disk(\bsc)^{\amalg} \times \disk(\bsc)^{\amalg} \arrow[r, "\simeq"] & (\disk(\bsc) \times \disk(\bsc))^{\amalg}  
        \end{tikzcd}.
    \end{equation}
    %
    Since $\disk_{/M \times N}$ is also given by a pullback, to provide an equivalence of $\infty$-operads $\disk_{/M} \times \disk_{/N} \xrightarrow{} \disk_{/M \times N}$ it is then sufficient to provide a commutative diagram of equivalences of $\infty$-operads
    %
    \begin{equation}
        \begin{tikzcd}
            \disk(\bsc) \times \disk(\bsc) \arrow[r] \arrow[d, "\simeq"] & (\disk(\bsc) \times \disk(\bsc))^{\amalg} \arrow[d, "\simeq"] & (\disk_{/M} \times \disk_{/N})^{\amalg} \arrow[l] \arrow[d, "\simeq"] \\
            \disk(\bsc) \arrow[r] & \disk(\bsc)^{\amalg} & (\disk_{/M \times N})^{\amalg} \arrow[l]
        \end{tikzcd}
    \end{equation}
    %
    Looking back at the construction \cite[Con.2.4.3.1]{lurie_ha} of $\mathscr{O}^{\coprod}$, since it is given by a universal property we see that an equivalence of $\infty$-categories $\mathscr{O}_1 \simeq \mathscr{O}_2$ implies an equivalence of $\infty$-operads $\mathscr{O}_1^{\amalg} \simeq \mathscr{O}_2^{\amalg}$ such that the left-hand square in the above diagram would come for free. With this the ingredients we need are:
    %
    \begin{enumerate}
        \item an equivalence of $\infty$-operads
        %
        \begin{equation}
            \disk(\bsc) \times \disk(\bsc) \xrightarrow[\ \ \simeq \ \ ]{\times} \disk(\bsc),
        \end{equation}
        %
        \item an equivalence, now only, of $\infty$-categories
        %
        \begin{equation}
            \disk_{/M} \times \disk_{/N} \xrightarrow[\ \ \simeq \ \ ]{\times} \disk_{/M \times N},
        \end{equation}
        %
        \item such that there is a commutative diagram of $\infty$-categories
        %
        \begin{equation}
            \begin{tikzcd}
                \disk_{/M} \times \disk_{/N} \arrow[r, "\times", "\simeq"'] \arrow[d] & \disk_{/M \times N} \arrow[d] \\
                \disk(\bsc) \times \disk(\bsc) \arrow[r, "\times", "\simeq"'] & \disk(\bsc)
            \end{tikzcd},
        \end{equation}
        %
        where we have forgotten the $\infty$-operad equivalence down to an $\infty$-category equivalence.
    \end{enumerate}
    %
    In the above we have already used the notation $\times$ to express an idea about what the maps we will provide are. Though there is abuse of notation it is clear from context which map $\times$ we want. Since the vertical maps are the canonical forgetful functors it will be clear from the construction that (3) is satisfied.

    To get the data of (2) (..... reduce to $\bsc$). So we need to provide an equivalence of $\infty$-categories
    %
    \begin{equation}\label{eq:b/M_x_b/N=b/MxN}
        \bsc_{/M} \times \bsc_{/N} \xrightarrow{\times} \bsc_{/M \times N}.
    \end{equation}

    To get (1), we remember that as an $\infty$-operad $\disk(\bsc)$ is, in fact, the free symmetric monoidal $\infty$-category generated by $\bsc$, thus it suffices to provide an equivalence of $\infty$-categories
    %
    \begin{equation}
        \bsc \times \bsc \xrightarrow[\ \ \simeq \ \ ]{\times} \bsc.
    \end{equation}
    %
    Remembering that $\bsc$ has a terminal object $*$ so that $\bsc_{/*} \simeq \bsc$, this equivalence is just a special case of \Cref{eq:b/M_x_b/N=b/MxN} in the case when $M = N = *$.

    The $\infty$-category $\bsc_{/M}$ is also known, by definition, as $\mathsf{Entr(M)}$, the enter-path $\infty$-category of the stratified manifold $M$. By \cite[Lem.3.3.9]{afr_homhyp} this is equivalent to the opposite $\infty$-category of the exit-path $\infty$-category of $M$. By \cite[Obs.3.3.3]{afr_homhyp} the equivalence
    %
    \begin{equation}
        \mathsf{Exit}(M) \times \mathsf{Exit}(N) \xrightarrow[\ \ \simeq \ \ ]{\times} \mathsf{Exit}(M \times N)
    \end{equation}
    %
    is immediate.

    To complete the proof we only have to check that $\times: \disk_{/M} \times \disk_{/N} \xrightarrow{} \disk_{M \times N}$ satisfies the conditions to qualify as a bifunctor of $\infty$-operads.

    ....

    The equivalence of $\infty$-operads $\disk_{/M} \otimes \disk_{/N} \xrightarrow{\ \ \simeq \ \ } \disk_{/M \times N}$ implies an equivalence at the level of $\alg{\disk_{/M}}$, which together with \Cref{thm:disk_alg=lcfa} gives rise to the equivalence of factorization algebras in the theorem statement. That this is related to the functor $\bar{\pi}$ can be seen by inspection of the objects and morphisms.
\end{proof}

\begin{remark}
    At the pointwise level of evaluating specific opens this statement is related to the Fubini theorem for factorization homology \cite[cor.2.29]{aft_fhstrat}. The notation used in the citation is technically abusive because factorization homology is evaluating lower dimensional manifolds, but it is clear from the context that this is to be understood exactly as the evaluation of the pushforward factorization algebra on the lower dimensional manifold (see \Cref{rem:fh_is_pushingforward}).
\end{remark}

\begin{example}
    We can view a Euclidean space $\mathbb{R}^{m+n}$ as a product space $\mathbb{R}^m \times \mathbb{R}^n$, in which case \Cref{thm:additivity_lc} specializes to give the famous result of Dunn additivity \cite{dunn1988} which can be found in the following form in \cite{lurie_ha}:
    %
    \begin{proposition}\label{prop:dunn_additivity}
        There is an equivalence of $\infty$-categories
        %
        \begin{equation}
            \alg{\mathbb{E}_{n + m}} (\catC) \xrightarrow{\ \ \simeq \ \ } \alg{\mathbb{E}_n} (\alg{\mathbb{E}_m} (\catC)).
        \end{equation}
    \end{proposition}
\end{example}

\begin{remark}
    The proof of \Cref{thm:additivity_lc}, when specialized to ordinary manifolds, as given in \cite{ginot2015} actually depends on Dunn additivity to work, however the proof presented above and the one in \cite{lurie_ha} are both independent of this special case. 
\end{remark}



\end{document}