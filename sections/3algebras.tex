\documentclass[../text]{subfiles} 

\begin{document}

\section{Factorization Algebras}\label{ch:fact_alg}

\subsection{Definition of Factorization Algebras}

Factorization algebras are a rigorous way to capture the idea of assigning an object to each piece of a space, together with a local-to-global principle that allows one to find out about larger pieces from smaller ones. The first ideas around factorization algebras were introduced by \cite{bd2004} in the context of vertex algebras in CFTs. These were geometric in nature. The topological version that we present was then introduced by Lurie and subsequently further developed in \cite{cg2016}, especially for the purposes of mathematical physics. Here we will present these ideas following \cite{cg2016} and also \cite{af_primer}. A similar exposition can be found in \cite{ginot2015}.

\begin{remark}
    We note that we have fixed a symmetric monoidal $\infty$-category $\catC$ that is $\otimes$-presentable to serve as the $\infty$-category that our algebras will be valued in.
\end{remark}

\begin{definition}
    Let $X$ be a topological space. We regard $\opens[s](X)$ as a multicategory (and consequently as an $\infty$-operad) whose objects are the open sets of $X$, and through the assignment of a unique multi-morphism from $\{U_i\}_{i \in I}$ to $U$ if the $U_i$ are pairwise disjoint and if $\bigcup_{i \in I} U_i \subset U$. A \emph{prefactorization algebra $F$ on $X$ with values in $\catC$} is an algebra in $\catC$ over this $\infty$-operad
    %
    \begin{equation}
        F \in \alg{\opens[s](X)}(\catC).
    \end{equation}
\end{definition}

\begin{remark}
    It is obvious that in the case of (stratified) manifolds we have already seen alternative notation for the $\infty$-operad $\opens(X)$ before, namely $\mfld[r]_{/M}$. This makes a connection to the standard notation used around factorization homology, and is something that we'll make use of later.
\end{remark}

\begin{remark}
    Let's unwind the above definition to understand the underlying data needed to define a prefactorization algebra. On the level of objects, for every open set $U$ we need to assign an object $F(U) \in \catC$. On the level of morphisms, given a pairwise disjoint set of opens $\{U_i\}_{i \in I}$ and an open $U$, such that the $U_i$ all lie in $U$ we need to assign a morphism in $\catC$
    %
    \begin{equation}
        \bigotimes_{i \in I} F(U_i) \xrightarrow{F(\{U_i\},U)} F(U).
    \end{equation}
    %
    As the notation suggests and the operadic symmetry condition guarantees, the maps $F(\{U_i\},U)$ only depend on the set of $U_i$ and not on any particular order of the $U_i$, which is also allowed on the left-hand side by the symmetric monoidal structure of $\catC$. Furthermore, the operadic associativity condition imposes that given pairwise disjoint $\{U_i\}_{i \in I}$ that all lie in $U$, and for each $i \in I$, pairwise disjoint $\{ V_{i, j}\}_{j \in J_i}$ that all lie in $U_i$, there is a commutative diagram in $\catC$
    %
    \begin{equation}
        \begin{tikzcd}
            \bigotimes\limits_{(i,j)} F(V_{i,j}) \arrow[rd, shorten <=-2ex, "\bigotimes\limits_{i} F(\{V_{i, j}\} {,} U_i)"'] \arrow[rr, "F( \{V_{i, j}\} {,} U)"] &   & F(U) \\
            & \bigotimes\limits_{i} F(U_i) \arrow[ru, "F( \{U_i \} {,} U)"'] &,          
        \end{tikzcd}
    \end{equation}
    %
    where the $(i,j)$ tensor product runs over all possible pairs $i \in I$ and $j \in J_i$. Finally, the operadic unitarity condition actually tells us that the morphisms $F(\{U\}, U)$ are equivalent to the identity morphism.

    We also get out that an algebra morphism $\phi: F \rightarrow G$ is simply a family of maps $\phi(U): F(U) \rightarrow G(U)$ for each open set $U$, such that it respects the operations of the algebra. Namely, for each multi-morphism $\{ U_i \}_{i \in I} \rightarrow U$ there is a commuting square
    %
    \begin{equation}\label{cd:fa_morphisms}
        \begin{tikzcd}[row sep=large, column sep = large]
            \bigotimes\limits_{i \in I} F(U_i) \arrow[r, "F(\{ U_i\} {,} U)"] \arrow[d, "\bigotimes\limits_{i \in I} \phi(U_i)"'] & F(U) \arrow[d, "\phi(U)"] \\
            \bigotimes\limits_{i \in I} G(U_i) \arrow[r, "G(\{ U_i\} {,} U)"'] & G(U).
        \end{tikzcd}
    \end{equation}

\end{remark}

\begin{remark}\label{rem:op_to_op}
    There is an obvious inclusion functor from the poset $\opens[r](X)$ of open subsets of $X$, ordered by inclusion, to the multicategory $\opens[s](X)$, namely, the one which hits only those multi-morphisms $\{U_i\}_{i \in I} \rightarrow U$, where the cardinality of the finite set $I$ is 1.
\end{remark}


A factorization algebra will be a prefactorization algebra that further satisfies a certain gluing condition that lets us construct its value on `larger' sets if we know it on `smaller' sets. More formally, this will be a kind of cosheaf condition, that we describe now, following \cite{weiss1999,af_primer}.

We endow the poset $\opens[r](X)$ with a Grothendieck topology called the \emph{Weiss} Grothendieck topology \cite{weiss1999}. In this topology a sieve $\mathscr{U} \subset \opens[r](X)_{/U}$ on $U$ is a covering sieve if for each finite subset $S \subset U$, there is an object $(e: V \rightarrow U) \in \mathscr{U}$ for which $S \subset e(V)$. In other words, a family $\{ V_i \rightarrow U\}_{i \in I}$ is a Weiss cover of $U$ if every set of finitely many points in $U$ is contained in some $V_i$. Contrast this with the standard Grothendieck topology on the poset of opens $\opens[r](X)$, in which instead of a finite set we have a one-element set. Thus, every Weiss cover is a cover in the standard sense, but not necessarily the other way around.

\begin{definition}
    The $\infty$-category of \emph{$\catC$-valued Weiss (homotopy) cosheaves on $X$} is the full $\infty$-subcategory
    %
    \begin{equation}
        \csheaves[W]_{X}(\catC) \subset \mathsf{Fun}(\opens[r](X), \catC),
    \end{equation}
    %
    of the $\infty$-category of copresheaves, consisting of those functors $F: \opens[r](X) \rightarrow \catC$ for which, for each Weiss covering sieve $\mathscr{U} \subset \opens[r](X)_{/U}$, the canonical functor
    %
    \begin{equation}
        \mathscr{U}^{\vartriangleright} \rightarrow \opens[r](X)_{/U} \rightarrow \opens[r](X) \xrightarrow{F} \catC,
    \end{equation}
    %
    where $\mathscr{U}^{\vartriangleright} \rightarrow \opens[r](X)_{/U}$ is the functor from the colimit cone that assigns $U$ to the colimit object, is a colimit diagram.
\end{definition}

\begin{remark}
    In other words, $F$ is a Weiss cosheaf if it is a functor that sends colimits (which, in this case, are unions) of Weiss covers $\{ U_i \xhookrightarrow{} U \}_{i \in I}$ to colimits in $\catC$
    %
    \begin{equation}
        F(U) \simeq F \bigg( \bigcup_{i \in I} U_i \bigg) \simeq \mathrm{colim}_{i \in I} F(U_i).
    \end{equation}
\end{remark}

\begin{remark}
    The category $\opens[r](X)$ can be replaced with any other category that supports an analogue of the Weiss Grothendieck topology. In the case of (stratified) manifolds, $\opens[r](X)$ is already the same as $\mfld[r]_{/X}$, be we could just as well define Weiss cosheaves on $\mfld[r]$, i.e. $\csheaves[W](\mfld[r], \catC)$. If we're working with the $\infty$-categories $\mfld$ or $\mfld_{/X}$, then we define cosheaves on them through the pullback
    %
    \begin{equation}
        \begin{tikzcd}
            \csheaves[W] (\mfld, \catC) \arrow[dr, phantom, "\ulcorner", very near start] \arrow[r] \arrow[d] & \mathsf{Fun}(\mfld, \catC) \arrow[d] \\
            \csheaves[W] (\mfld[r], \catC) \arrow[r] & \mathsf{Fun}(\mfld[r], \catC),
        \end{tikzcd}
    \end{equation}
    %
    and similarly for the relative case $\mfld_{/X}$.
\end{remark}

\begin{definition}
    The $\infty$-category of \emph{($\catC$-valued) factorization algebras on $X$}\footnote{These are called homotopy factorization algebras in \cite{cg2016}, however we do not consider the lax version of factorization algebras that they also present.} is the full $\infty$-subcategory of $\alg{\opens[s](X)}$ as in the pullback
    %
    \begin{equation}
        \begin{tikzcd}
            \falg_X (\catC) \arrow[dr, phantom, "\ulcorner", very near start] \arrow[r] \arrow[d] & \alg{\opens[s](X)}(\catC) \arrow[d] \\
            \csheaves[W]_X(\catC) \arrow[r] & \mathsf{Fun}(\opens[r](X), \catC).
        \end{tikzcd}
    \end{equation}
\end{definition}

\begin{proof}
    The previous definition implies that the top horizontal functor is fully faithful, which is something we need to prove. The bottom horizontal functor is fully faithful, by the definition of cosheaf, and the right vertical functor forgets the operations with arity higher than $1$. The key observation is that a Weiss cover is pairwise disjoint only when it consists of a single subset. Thus, the cosheaf condition does not affect the algebra morphisms and an equivalence of morphism spaces is trivial to find, namely one is given by the identity map.
\end{proof}

\begin{remark}\label{rem:check_nerve}
    A factorization algebra is a prefactorization algebra whose restriction to $\opens[r](X)$ is a Weiss cosheaf. This means that the definition can be rewritten in an equivalent way as with all cosheaves. Following \cite{nlab} (which presents the dual case of sheaves) we can express the condition through the \v{C}ech nerve. Let $\mathscr{U}_U = \{ U_i \rightarrow U\}_{i \in I}$ be a (Weiss) cover of an open set $U$ and $F$ be a copresheaf, and define the simplicial object
    %
    \begin{equation}
        \cech_{\bullet}(\mathscr{U}_U, F) := \left( \coprod_{i \in I} F(U_i) \leftleftarrows \coprod_{i, j \in I} F(U_i \cap U_j) \mathrel{\substack{\textstyle\leftarrow\\[-0.6ex]
        \textstyle\leftarrow\\[-0.6ex]\textstyle\leftarrow}} \coprod_{i, j, k \in I} F(U_i \cap U_j \cap U_k) \mathrel{\substack{\textstyle\leftarrow\\[-0.6ex]\textstyle\leftarrow\\[-0.6ex]\textstyle\leftarrow\\[-0.6ex]\textstyle\leftarrow}} \cdots  \right),
    \end{equation}
    %
    where the maps are induced by the value of $F$ on the inclusions of opens. There are canonical maps $F(U_i) \rightarrow F(U)$, so that at the level of simplicial objects there is a canonical map
    %
    \begin{equation}
        \cech_{\bullet} (\mathscr{U}_U, F) \xrightarrow{\quad} F(U)_{\bullet},
    \end{equation}
    %
    to the constant simplicial complex at $F(U)$. Since $\catC$ has (sifted) colimits we can take the geometric realization to obtain a morphism
    %
    \begin{equation}
        \cech(\mathscr{U}_U, F) \xrightarrow{\quad} F(U).
    \end{equation}
    %
    $F$ is a (homotopy) cosheaf if, and only if, this morphism is an equivalence for every open set $U$ and every Weiss cover of it $\mathscr{U}_U$.
\end{remark}

All of our definitions above are valid for all topological spaces, so in particular, also the stratified spaces we introduced at the beginning. For the next definition of locally constant factorization algebras though, it will be important to specify more information about the space. In other words, they will capture information about the stratification.

There is a general definition of locally constant algebra objects in the context of any $\infty$-operad given in \cite[def.2.3.3.20]{lurie_ha}, which, of course, applies here. In our case one can also take a perspective more akin to \cite[def.A.1.12]{lurie_ha} by defining locally constant cosheaves and then pulling back this property to factorization algebras. This is what we will do, along with focusing our attention to stratified (and therefore also to smooth) manifolds. The definitions we will make can in a lot of cases also be expanded for any $C^0$ stratified space or topological manifold.

\begin{definition}
    The $\infty$-category of locally constant Weiss cosheaves on a stratified manifold $M$ (or, equivalently, on $\mfld[r]_{/M}$) is the full $\infty$-subcategory
    %
    \begin{equation}
        \csheaves[W, lc]_M (\catC) \subset \csheaves[W]_M (\catC),
    \end{equation}
    %
    of those Weiss cosheaves whose underlying functor $F: \bsc_{/M} \xrightarrow{} \catC$ satisfies
    %
    \begin{equation}
        f(U) \simeq f(V) \Rightarrow F(U) \simeq F(V),
    \end{equation}
    %
    for all $U, V \in \bsc_{/M}$, where $f: \bsc_{/M} \xrightarrow{} \bsc$ denotes the forgetful functor.
\end{definition}

\begin{remark}
    Thus, a functor is locally constant if it takes open embeddings of basics $V \xhookrightarrow{} U$, with equivalent stratifications to equivalences in $\catC$. We emphasize this last point about stratification; the disks have to have equivalent stratifications inherited from the stratified space. In the case of $\hoint$, for example, open embeddings like $(a, b) \rightarrow \left[0, c\right)$, for $0 \leq a < b \leq c$, would \emph{not} be taken to equivalences since the disks have different stratifications.
\end{remark}

\begin{remark}
    Since being locally constant according to the previous definition is only a property of the underlying functor the definition can easily be used to define locally constant prefactorization algebras too.
\end{remark}

\begin{definition}
    The $\infty$-category $\lcfa_M (\catC)$ of \emph{locally constant factorization algebras on a stratified manifold $M$ valued in $\catC$} is the full $\infty$-subcategory of $\falg_M (\catC)$ as in the pullback
    %
    \begin{equation}
        \begin{tikzcd}
            \lcfa_M (\catC) \arrow[dr, phantom, "\ulcorner", very near start] \arrow[r] \arrow[d] & \falg_M (\catC) \arrow[d] \\
            \csheaves[W, lc]_M (\catC) \arrow[r] & \csheaves[W]_M (\catC).
        \end{tikzcd}
    \end{equation}
\end{definition}

\begin{remark}\label{rem:sym_mon_inheritance}
    The $\infty$-categories of factorization algebras of all varieties acquire a symmetric monoidal structure from the symmetric monoidal structure of $\catC$ pointwise
    %
    \begin{equation}
        (F \otimes G)(U) = F(U) \otimes G(U).
    \end{equation}
    %
    The only subtlety is the cosheaf condition, where the symmetric monoidal structure is induced only if we also take into account that in $\catC$ colimits commute with its symmetric monoidal structure.

    The existence of colimits is also inherited from $\catC$ pointwise through the collection of evaluation functors $\{ \mathsf{ev}_U: \falg_X (\catC) \rightarrow \catC \}$, one for each open $U \subset X$.
\end{remark}



\subsection{The \texorpdfstring{$\infty$-}{infinity }operad of Factorization Algebras}

Now that we have introduced all the varieties of factorization algebras, it's a natural and useful question to ask what the $\infty$-operad governing them is. For prefactorization algebras, the answer comes by definition, but this is not the case for (locally constant) factorization algebras. We will be working in the generality of stratified manifolds, and all our results in this section will come in pairs, one for the locally constant case and one not. We call the results in the not-necessarily-locally-constant case corollaries since their proofs will be obvious from the proofs of the statements in the locally constant case. To connect factorization algebras to the theory of algebras over an $\infty$-operad we will work in two steps. First we have the following results:
%
\begin{proposition}\label{prop:fh_disk_to_lcfa}
    Given a stratified manifold $M$ factorization homology provides a functor
    %
    \begin{equation}
        \int : \alg{\disk_{/M}} (\catC) \xrightarrow{\ \ \ \ } \lcfa_M (\catC)
    \end{equation}
    %
    between the $\infty$-categories of $\disk_{/M}$-algebras and locally constant factorization algebras $\lcfa_M$.
\end{proposition}
%
\begin{corollary}\label{cor:fh_disk_to_falg}
    Given a stratified manifold $M$ factorization homology provides a functor
    %
    \begin{equation}
        \int: \alg{\disk[r]_{/M}} (\catC) \xrightarrow{\ \ \ \ } \falg_{M} (\catC).
    \end{equation}
\end{corollary}

The second step is then showing that the functors constructed above are equivalences, which we summarize in the following results:
%
\begin{theorem}\label{thm:disk_alg=lcfa}
    For each stratified manifold $M$, the previously constructed factorization homology functor provides an equivalence of $\infty$-categories
    %
    \begin{equation}
        \alg{\disk_{/M}} (\catC) \xrightarrow[\int]{\ \ \simeq \ \ } \lcfa_M (\catC)
    \end{equation}
    %
    between $\disk_{/M}$-algebras and locally constant factorization algebras on $M$. Equivalently, the $\infty$-operad governing locally constant factorization algebras can be taken to be $\disk_{/M}$.
\end{theorem}
%
\begin{corollary}\label{cor:disk_alg=falg}
    The $\infty$-operad governing factorization algebras on a stratified manifold $M$ can be taken to be $\disk[r]_{/M}$,
    %
    \begin{equation}
        \alg{\disk[r]_{/M}} (\catC) \xrightarrow[\int]{\ \ \simeq \ \ } \falg_M (\catC).
    \end{equation}
\end{corollary}
%
The reason for the phrasing in these results is that, technically speaking, equivalence of the $\infty$-categories of algebras does not imply equivalence of the underlying operads. However since we only defined (locally constant) factorization algebras at the level of algebras and not on the level of $\infty$-operads, we can take the operadic definition as the basic one, and interpret the results as connecting this to the classical definition. The theorems that we present in this section are immensely helpful in relating factorization algebras to results in the area of disk algebras, which will be obvious throughout the rest of this work. One important example that we will see is in \Cref{ch:gluing_disk_alg}.

In the service of working towards proofs of all of these results we first have to pause and develop some necessary technically results regarding Weiss covers and preservation of colimits.

\begin{lemma}\label{lem:weiss_to_ordinary}
    Let $M$ be a topological space with a sieve $\mathscr{U}$. The following are equivalent:
    %
    \begin{enumerate}
        \item $\mathscr{U}$ is a Weiss sieve of $M$,
        \item for all $n \geq 0$, the smallest sieve containing $\{ \mathrm{Conf}_n(U)_{\Sigma_n} \xhookrightarrow{} \mathrm{Conf}_n(M)_{\Sigma_n} \mid U \in \mathscr{U} \}$ is a covering sieve of the unordered configuration space $\mathrm{Conf}_n(M)_{\Sigma_n}$ of $n$ points in $M$.
    \end{enumerate}
\end{lemma}

\begin{proof}
    In the forward direction, let $[(x_1, \dots, x_n)] \in \mathrm{Conf}_n(M)_{\Sigma_n}$ be an equivalence class of $n$ points under the action of the symmetric group. By definition of Weiss sieve, there is a $U \in \mathscr{U}$ such that $\{x_1, \dots, x_n\} \subset U$. Since none of the points are the same this means that $(x_1, \dots, x_n) \in U^{\times n}$, and subsequently that $(x_1, \dots, x_n) \in \mathrm{Conf}_n(U)$. Taking quotients we get that $[(x_1, \dots, x_n)] \in \mathrm{Conf}_n(U)_{\Sigma_n}$. Since $\mathrm{Conf}_n(U)_{\Sigma_n}$ is open in the topology of $\mathrm{Conf}_n(M)_{\Sigma_n}$, the collection of these form a covering sieve.

    In the opposite direction, let $S \subset M$ be a finite set with cardinality $|S| = n$. Naming its elements $x_1, \dots, x_n$ in an arbitrary order that will not matter because of the $\Sigma_n$ quotient, we get the $[(x_1, \dots, x_n)] \in \mathrm{Conf}_n(M)_{\Sigma_n}$. By assumption there exists a $U \in \mathscr{U}$ such that $[(x_1, \dots, x_n)] \in \mathrm{Conf}_n(U)_{\Sigma_n}$. In particular, this means that $(x_1, \dots, x_n) \in U^{\times n}$, or in other words that $x_i \in U$ for all $1 \leq i \leq n$. This implies that $\mathscr{U}$ satisfies the Weiss property.
\end{proof}

%To show the equivalence, we follow a related strategy to the one laid out in part 2 of the proof of \cite[Prop.2.22]{af_primer}.


\begin{lemma}\label{lem:gluing_disk/M/e}
    A Weiss sieve $\mathscr{U} \subset (\mfld[r]_{/M})_{/e}$ canonically determines a morphism
    %
    \begin{equation}
        \mathsf{colim} \big( \mathscr{U} \xrightarrow{} \mfld[r]_{/M} \xrightarrow{\mathsf{Hom}_{\mfld_{/M}}(\cdot, -)} \mathsf{PShv}(\disk_{/M}) \big) \xrightarrow{\ \ \simeq \ \ } \mathsf{Hom}_{\mfld_{/M}}(\cdot, e)
    \end{equation}
    of presheaves over $\disk_{/M}$, which moreover, is an equivalence.
\end{lemma}

\begin{proof}
    The morphism is determined by the universal property of the colimit. The components of the morphism of presheaves are given by
    %
    \begin{equation}
        \mathsf{colim} \big( \mathscr{U} \xrightarrow{} \mfld[r]_{/M} \xrightarrow{\mathsf{Hom}_{\mfld_{/M}}(\iota, -)} \mathsf{Sp} \big) \xrightarrow{\ \ \ \ \ } \mathsf{Hom}_{\mfld_{/M}}(\iota, e),
    \end{equation}
    %
    for each $\iota \in \disk_{/M}$, an open embedding of disks in $M$, and these morphisms are now maps of spaces which we want to show are equivalences. We now study the hom-spaces appearing above. By definition of the slice $\infty$-category, the hom-spaces are given as the homotopy fibers
    %
    \begin{equation}
        \begin{tikzcd}
            \mathsf{Hom}_{\mfld_{/M}} (\iota, e) \arrow[d] \arrow[r] & \mathsf{Hom}_{\mfld}(D, N) \arrow[d, "e^*"] \\
            * \arrow[r, "\{\iota\}"] & \mathsf{Hom}_{\mfld}(D, M),
        \end{tikzcd}
    \end{equation}
    %
    where we have fixed the notation $\iota: D \xhookrightarrow{} M$ and $e: N \xhookrightarrow{} M$, for the domains of these maps. We further establish the notation that for each isomorphism class of basics $[B]$, the fixed $D$ has $i_[B]$-many disks of type $[B]$. With this notation in hand, by the characterization of the maximal $\infty$-subgroupoid of $\disk_{/M}$ \cite[Lem.2.21]{aft_fhstrat} for example, we know that
    %
    \begin{equation}
        \mathsf{Hom}_{\mfld}(D, M) \simeq \coprod_{[B]} \mathrm{UConf}_{i_{[B]}}(M_{[B]}),
    \end{equation}
    %
    where the notation $M_{[B]}$ denotes the $[B]$-stratum of $M$ (see \cite[Prop.4.4.7]{aft_localstrut}). We also note that under these equivalences the map $e^*$ is given by $\coprod_{[B]} \mathrm{UConf}_{i_{[B]}}(e_{[B]})$, which is even an open inclusion. Thus, the hom-spaces of the slice $\infty$-category are given as 
    %
    \begin{equation}
        \mathsf{Hom}_{\mfld_{/M}} (\iota, e) \simeq \coprod_{[B]} \mathsf{Fib}_\iota^{[B]}(e),
    \end{equation}
    %
    where $\mathsf{Fib}_\iota^{[B]}(e)$ is defined as the homotopy fiber
    %
    \begin{equation}
        \begin{tikzcd}
            \mathsf{Fib}_\iota^{[B]}(e) \arrow[d] \arrow[r] & \mathrm{UConf}_{i_{[B]}}(N_{[B]}) \arrow[d, "\mathrm{UConf}_{i_{[B]}}(e_{[B]})"] \\
            * \arrow[r, "\{\iota_{[B]}\}"] & \mathrm{UConf}_{i_{[B]}}(M_{[B]}),
        \end{tikzcd}
    \end{equation}
    %
    where, by slight abuse, we denoted the point over which the fiber lies with $\iota_{[B]}$ as well. With all of this notation in place, and by switching the places of the colimits, what we want to show is that the canonical maps
    %
    \begin{equation}
        \mathsf{colim} \big( \mathscr{U} \xrightarrow{} \mfld[r]_{/M} \xrightarrow{\mathsf{Fib}_{\iota}^{[B]}(-)} \mathsf{Sp} \big) \xrightarrow{\ \ \ \ \ } \mathsf{Fib}_{\iota}^{[B]}(e),
    \end{equation}
    %
    are equivalences for every $\iota$ and every $[B]$. Using \cite[Prop.A.3.2]{lurie_ha}, a sufficient condition for these maps to be equivalences is for the composition $\mathscr{U} \xrightarrow{} \mfld[r]_{/M} \xrightarrow{\mathsf{Fib}_{\iota}^{[B]}(-)} \mathsf{Sp}$ to be a covering sieve of $\mathsf{Fib}_{\iota}^{[B]}(e)$. To exhibit such a property we pick a particular model for the homotopy fiber $\mathsf{Fib}_{\iota}^{[B]}(e)$, namely
    %
    \begin{equation}
        \widetilde{\mathsf{Fib}_{\iota}^{[B]}}(N) := \big( \mathsf{P}(\mathrm{UConf}_{i_{[B]}}(M_{[B]})) \times \mathrm{UConf}_{i_{[B]}}(N_{[B]}) \big) / (s (p) = \iota_{[B]}, t (p) = e(x)),
    \end{equation}
    %
    the space of paths in $\mathrm{UConf}_{i_{[B]}}(M_{[B]})$ that start at $\iota_{[B]}$ and end at a point $x$ in $\mathrm{UConf}_{i_{[B]}}(N_{[B]})$. With this model, for each $(U \xhookrightarrow{} N) \in \mathscr{U} \subset (\mfld[r]_{/M})_{/e} \simeq \mfld[r]_{/N}$ (by \Cref{lem:double_slice}) we get an open inclusion
    %
    \begin{equation}
        \widetilde{\mathsf{Fib}_{\iota}^{[B]}}(U) \xhookrightarrow{\ \ \ \ } \widetilde{\mathsf{Fib}_{\iota}^{[B]}}(N),
    \end{equation}
    %
    and we're trying to show that the collection $\mathscr{V}_{\iota}^{[B]}$ of these is a cover.

    Given a Weiss cover $\mathscr{U}$ on $N$ we can restrict it to a Weiss cover $\mathscr{U}_{[B]}$ of $N_{[B]}$ by intersecting. The statement of \Cref{lem:weiss_to_ordinary} is that we can build an ordinary cover $\mathscr{U}_{i_{[B]},[B]}$ of $\mathrm{UConf}_{i_{[B]}}(N_{[B]})$. $\mathscr{V}_{\iota}^{[B]}$ is a cover if for any $p \in \widetilde{\mathsf{Fib}_{\iota}^{[B]}}(N)$ we can find an open $\widetilde{\mathsf{Fib}_{\iota}^{[B]}}(U)$ that contains it. But this is indeed the case, because looking at the target $t(p)$ of the path $p$, we get $t(p) \in \mathrm{UConf}_{i_{[B]}}(N_{[B]})$, and we know that there exists a $U \in \mathscr{U}$ such that $\mathrm{UConf}_{i_{[B]}}(U_{[B]})$ covers $t(p)$ and subsequently also that $\widetilde{\mathsf{Fib}_{\iota}^{[B]}}(U)$ covers $p$.

    ...
\end{proof}


The following lemma about preservation of colimits is sketched out in \cite{rezk2023lan}.
%
\begin{lemma}\label{lem:Lan_preserve_colim}
    Let $F: \mathscr{D} \xrightarrow{} \catC$ and $\iota: \mathscr{D} \xrightarrow{} \mathscr{M}$ be functors of (locally small) $\infty$-categories, where $\mathscr{M}$ and $\catC$ are cocomplete. Let $L_{\iota} F: \mathscr{M} \xrightarrow{} \catC$ be the left Kan extension of $F$ along $\iota$. $L_{\iota} F$ preserves a given colimit if, and only if, the restricted Yoneda embedding
    %
    \begin{align}
        T_{\iota}: \mathscr{M} \xrightarrow{\ y_{\mathscr{M}} \ } \mathsf{PShv}(\mathscr{M}) \xrightarrow{\ \iota^* \ } \mathsf{PShv}(\mathscr{D}) 
    \end{align}
    %
    does so.
\end{lemma}

\begin{proof}
    First consider the special case when $(\iota: \mathscr{D} \xrightarrow{} \mathscr{M}) = (y_{\mathscr{D}}: \mathscr{D} \xrightarrow{} \mathsf{PShv}(\mathscr{D}))$ the Yoneda embedding of $\mathscr{D}$. We will see that in this case, $L_{y_{\mathscr{D}}} F$ is colimit preserving because it is a left adjoint to the functor
    %
    \begin{equation}
        T_F: \catC \xrightarrow{y_{\catC}} \mathsf{PShv} (\catC) \xrightarrow{F^*} \mathsf{PShv} (\mathscr{D}).
    \end{equation}
    %
    Indeed, let $X \simeq y_{\mathscr{D}} (d)$ be a representable presheaf, then there are maps of morphism spaces
    %
    \begin{align}
        \mathsf{Hom}_{\catC} (L_{y_{\mathscr{D}}} F (X), c) &\xrightarrow[\simeq]{} \mathsf{Hom}_{\catC} (F (d), c) \xrightarrow[\simeq]{y_{\mathscr{C}}} \mathsf{Hom}_{\mathsf{PShv}(\catC)} (y_{\catC}(F(d)), y_{\catC}(c))\notag\\ &\xrightarrow{F^*} \mathsf{Hom}_{\mathsf{PShv}(\mathscr{D})} (F^*(y_{\catC}(F(d))), T_F(c)) \xrightarrow[\simeq]{} \mathsf{Hom}_{\mathsf{PShv}(\mathscr{D})} (X, T_F(c)).
    \end{align}
    %
    The Yoneda embedding $y_{\mathscr{D}}$ is a dense functor, so the above argument holds  
    
    {\color{red} Proof of being adjoint}

    In the general situation where $\iota$ is not necessarily the Yoneda embedding we can partially reduce to the previous case because we can write
    %
    \begin{equation}
        L_{\iota} F \simeq L_{y_{\mathscr{D}}}F \circ L_{\iota} y_{\mathscr{D}}.
    \end{equation}
    %
    Namely, this is because we know that $L_{y_{\mathscr{D}}} F$ preserves colimits, and by the assumptions in the lemma, $L_{\iota} F$ and $L_{\iota} y_{\mathscr{D}}$ are computed pointwise, as colimits over the same indexing $\infty$-category.

    Finally, we claim that there is an equivalence of functors
    %
    \begin{equation}
        \Big( L_{\iota} y_{\mathscr{D}} \simeq T_{\iota} \Big) \in \mathsf{Fun}(\mathscr{M}, \mathsf{PShv}(\mathscr{D})) 
    \end{equation}

    {\color{red} Proof of functor equivalence}
    
\end{proof}













\begin{proof}[Proof of {\Cref{prop:fh_disk_to_lcfa}}]
    We adopt a proof strategy similar to \cite[prop.3.14]{af_primer}. Unlike there, we work in the relative case of $\disk_{/M}$ instead of $\disk(\bstr)$. Since the $\infty$-category of factorization algebras is defined as a pullback, we will look for a commutative diagram involving $\alg{\disk_{/M}}$, so that, by the universal property of the pullback, we get the stated functor.
    
    We first observe that, by definition, the $\infty$-operad $\opens(M) := \mfld[r]_{/M}$. There are clear functors of $\infty$-operads
    %
    \begin{equation}
        \disk_{/M} \xhookrightarrow{\ \ \ } \mfld_{/M} \xleftarrow{\ \ \ } \mfld[r]_{/M}.
    \end{equation}
    %
    At the level of algebras over them, they give rise to the commutative diagram
    %
    \begin{equation}
        \begin{tikzcd}
            \alg{\disk_{/M}} (\catC) \arrow[d] & \alg{\mfld_{/M}} (\catC) \arrow[l] \arrow[r] \arrow[d] & \alg{\opens(M)} (\catC) \arrow[d] \\
            \mathsf{Fun}(\disk_{/M}, \catC) & \mathsf{Fun}(\mfld_{/M}, \catC) \arrow[l] \arrow[r] & \mathsf{Fun}(\opens(M), \catC),
        \end{tikzcd}
    \end{equation}
    %
    where the vertical arrows are forgetful functors. Factorization homology as a left Kan extension gives one the adjoints indicated below
    %
    \begin{equation}
        \begin{tikzcd}[row sep = large, column sep = large]
            \alg{\disk_{/M}} (\catC) \arrow[r, bend left=10, "\int"] \arrow[d] & \alg{\mfld_{/M}} (\catC) \arrow[l, bend left=10] \arrow[r] \arrow[d] & \alg{\opens(M)} (\catC) \arrow[d] \\
            \mathsf{Fun}(\disk_{/M}, \catC) \arrow[r, bend left=10, "\int"] & \mathsf{Fun}(\mfld_{/M}, \catC) \arrow[l, bend left=10] \arrow[r] & \mathsf{Fun}(\opens(M), \catC).
        \end{tikzcd}
    \end{equation}
    %
    A result found in \cite[thm.1.2.5]{aft_localstrut} (where they use slightly different notation and call locally constant sheaves constructible) gives us the equivalence
    %
    \begin{equation}
        \csheaves[W](\mfld_{/M}, \catC) \simeq \csheaves[W, lc](\mfld[r]_{/M}, \catC) =: \csheaves[W, lc]_M(\catC),
    \end{equation}
    %
    which at the level of underlying functors, is exactly the restriction along the functor $\opens(M) \xrightarrow{} \mfld_{/M}$, so that we can append this to our commutative diagram
    %
    \begin{equation}
        \begin{tikzcd}[row sep = large, column sep = large]
            \alg{\disk_{/M}} (\catC) \arrow[r, bend left=10, "\int"] \arrow[d] & \alg{\mfld_{/M}} (\catC) \arrow[l, bend left=10] \arrow[r] \arrow[d] & \alg{\opens(M)} (\catC) \arrow[d] \\
            \mathsf{Fun}(\disk_{/M}, \catC) \arrow[r, bend left=10, "\int"] \arrow[dr, dashed] & \mathsf{Fun}(\mfld_{/M}, \catC) \arrow[l, bend left=10] \arrow[r] & \mathsf{Fun}(\opens(M), \catC) \\
            & \csheaves[W](\mfld_{/M}, \catC) \arrow[r, "\simeq"] \arrow[u] & \csheaves[W, lc]_M(\catC) \arrow[u].
        \end{tikzcd}
    \end{equation}
    %
    If we can find the dashed functor which makes the diagram commute, as indicated above, we would be done with the construction. This amounts to showing that the left Kan extension $\int F$ of a functor $F \in \mathsf{Fun}(\disk_{/M}, \catC)$ along $\disk_{/M} \xhookrightarrow{} \mfld_{/M}$ is automatically a Weiss cosheaf. By the definition of $\csheaves[W](\mfld_{/M}, \catC)$, this means that the restriction of $\int F$ to $\mathsf{Fun}(\mfld[r]_{/M}, \catC)$ satisfies the cosheaf property for every Weiss sieve $\mathscr{U} \subset (\mfld[r]_{/M})_{/e}$, where $e:N \xhookrightarrow{} M$ is an open embedding. Here we notice that for the Weiss property to even make sense, we are implicitly using \Cref{lem:double_slice}, so that the Weiss sieve $\mathscr{U}$ is defined as an $\infty$-subcategory of $\mfld[r]_{/N}$.
    
    With the Weiss cosheaf property being a statement about preserving colimits, and with $\int F$ being a left Kan extension we are exactly in the situation of \Cref{lem:Lan_preserve_colim}. Thus, we only need to show that the functor $T: \mfld_{/M} \xrightarrow{} \mathsf{PShv} (\disk_{/M})$ preserves Weiss colimits. Namely, we need to show that there is an equivalence
    %
    \begin{equation}
        \mathsf{colim} \big( \mathscr{U} \xrightarrow{} \mfld[r]_{/M} \xrightarrow{\mathsf{Hom}_{\mfld_{/M}}(\cdot, -)} \mathsf{PShv} (\disk_{/M}) \big) \xrightarrow{\ \ \simeq \ \ } \mathsf{Hom}_{\mfld_{/M}}(\cdot, e),
    \end{equation}
    %
    where on the right-hand side we've used the fact that $\mathscr{U}$ is, in particular, a covering sieve.\footnote{If we also append the straightening--unstraightening construction to $T$, the desired equivalence can equivalently be written as
    %
    \begin{equation}
        \mathsf{colim} (\mathscr{U} \xrightarrow{} \mfld_{/M} \xrightarrow{ (\disk_{/M})_{/-}} \mathsf{RFib}_{\disk_{/M}}) \simeq (\disk_{/M})_{/e}.
    \end{equation}}
    %
    But this is exactly the result of \Cref{lem:gluing_disk/M/e}.
\end{proof}

\begin{remark}
    Intuitively, the key idea for the statement and proof is that $\disk[r]_{/N}$ is a basis for the Weiss Grothendieck topology on manifolds. This is essentially because for each finite set of points $S \subset N$ we can take disks $\{ D_s \}_{s \in S}$ around each point, by virtue of $N$ being a manifold. Subsequently, for each $U_S$ that covers $S$ we can take small enough disks, whose disjoint union will cover $S$ by construction, as well as satisfy $\sqcup_{s \in S} D_s \subset U_S$. The proof makes this idea precise, and tells us that we can calculate factorization homology using any Weiss sieve
    %
    \begin{equation}
        \int_M F \simeq \mathsf{colim}(\disk[r]_{/M} \xrightarrow{} \disk_{/M} \xrightarrow{F} \catC)
        \simeq \mathsf{colim}(\mathscr{U} \xrightarrow{} \disk_{/M} \xrightarrow{F} \catC).
    \end{equation}
\end{remark}

The above proof also shows how the ideas that went into defining factorization algebras are not that different from the ideas that went into defining factorization homology. Namely, the ideas of Weiss cosheaves and algebras over disks are deeply connected. Now that this connection has been explained, the proof of \Cref{thm:disk_alg=lcfa}, which we delve into promptly, is not hard to establish.


\begin{proof}[Proof of {\Cref{thm:disk_alg=lcfa}}]
    To show essential surjectivity of $\int$, for each $F \in \lcfa_M (\catC)$ we will find an equivalence $\int F| \simeq F$ in $\lcfa_M (\catC)$, with $F|$ to be defined later. Since $\lcfa_M (\catC) \xrightarrow{} \alg{\opens(M)}$ is fully faithful, this amounts to finding an equivalence of prefactorization algebras, namely a family of equivalences
    %
    \begin{equation}
        \int_U F| \xrightarrow[\simeq]{\ \ \phi (U) \ \ } F(U),
    \end{equation}
    %
    one for each $(U \xhookrightarrow{} M) \in \mfld[r]_{/M}$, which are natural in $U$ and preserve the multiplicative structure (see \Cref{cd:fa_morphisms}). Since $F$ is a locally constant factorization algebra the functor
    %
    \begin{equation}
        \lcfa_M (\catC) \xrightarrow{} \csheaves[W, lc](\mfld[r]_{/M}, \catC) \simeq \csheaves[W](\mfld_{/M}, \catC) \xrightarrow{} \mathsf{Fun}(\mfld_{/M}, \catC),
    \end{equation}
    %
    which we omit in the notation, allows us to consider $F$ as a functor from $\mfld_{/M}$. Furthermore, since $\disk[r]_{/U}$ is a Weiss sieve we can write
    %
    \begin{align}
        F(U) &\simeq \mathsf{colim} (\disk[r]_{/U} \xrightarrow{} \disk[r]_{/M} \xrightarrow{} \mfld[r]_{/M} \xrightarrow{} \mfld_{/M} \xrightarrow{F} \catC).
    \end{align}
    %
    By consulting the commutative diagram
    % 
    \begin{equation}
        \begin{tikzcd}
            \disk[r]_{/U} \arrow[d] \arrow[r] & \disk_{/U} \arrow[d] & \\
            \disk[r]_{/M} \arrow[d] \arrow[r] & \disk_{/M} \arrow[d] \arrow[r, "F|"] & \catC, \\
            \mfld[r]_{/M} \arrow[r] & \mfld_{/M} \arrow[ru, "F"'] &  
        \end{tikzcd}
    \end{equation}
    %
    where $F|$ is the restriction of $F$ to disks, and using the fact that $\disk[r]_{/U} \xrightarrow{} \disk_{/U}$ is final (\cite[prop.2.22]{aft_fhstrat}) immediately gives us that
    %
    \begin{equation}
        F(U) \simeq \mathsf{colim} (\disk_{/U} \xrightarrow{} \disk_{/M} \xrightarrow{F|} \catC) \simeq \int_U F|.
    \end{equation}
    %
    Naturality of these equivalences is immediate from the construction because, as already used, each inclusion of open subsets $V \xhookrightarrow{} U$ gives a full subcategory inclusion $\disk[r]_{/V} \xhookrightarrow{} \disk[r]_{/U}$. Given this fact, to preserve the multiplicative structure it is enough to show that the following is a commutative diagram
    %
    \begin{equation}\label{eq:mult_strut_comm_square}
        \begin{tikzcd}
            \otimes_i \int_{U_i} F| \arrow[r, "\simeq"] \arrow[d, "\otimes_i \phi (U_i)"', "\simeq"] & \int_{\sqcup_i U_i} F| \arrow[d, "\phi (\sqcup_i U_i)", "\simeq"']\\
            \otimes_i F (U_i) \arrow[r, "\simeq"] & F (\sqcup_i U_i),
        \end{tikzcd}
    \end{equation}
    %
    for each collection of disjoint open sets $\{U_i\}$. This, however, is guaranteed by the equivalences
    %
    \begin{equation}
        \disk[r]_{/\sqcup_i U_i} \xrightarrow{\ \ \simeq \ \ } \bigtimes_i \disk[r]_{/U_i}.
    \end{equation}
    %
    Indeed, the fact that (sifted) colimits commute with the monoidal product on $\catC$ means that, for example in the case of two objects
    %
    \begin{align}
        \mathsf{colim}(K \xrightarrow{F} \catC) \otimes \mathsf{colim}(K' \xrightarrow{F'} \catC) &\simeq \mathsf{colim}(K \xrightarrow{F(-) \otimes \mathsf{colim}(K' \xrightarrow{F'} \catC)} \catC)\notag\\
        &\simeq \mathsf{colim}(K \xrightarrow{\mathsf{colim}(K' \xrightarrow{F(-) \otimes F'(-)} \catC)} \catC)\notag\\
        &\simeq \mathsf{colim}(K \times K' \xrightarrow{F \times F'} \catC \times \catC \xrightarrow{- \otimes -} \catC),
    \end{align}
    %
    for any (sifted) indexing $\infty$-categories $K, K'$.

    To show full faithfulness we need to find equivalences of morphism spaces
    %
    \begin{equation}
        \mathsf{Hom}_{\alg{\disk_{/M}} (\catC)} (A, B) \simeq \mathsf{Hom}_{\lcfa_M (\catC)} (\int A, \int B),
    \end{equation}
    %
    for each $A, B \in \alg{\disk_{/M}}$. However, since the functors
    %
    \begin{align}
        &\alg{\disk_{/M}} (\catC) \xrightarrow{\mathsf{f.f.}} \alg{\disk[r]_{/M}} (\catC) &\lcfa_M (\catC) \xrightarrow{\mathsf{f.f.}} \alg{\mfld[r]_{/M}} (\catC)
    \end{align}
    %
    are both fully faithful we are reduced to finding equivalences
    %
    \begin{equation}
        \mathsf{Hom}_{\alg{\disk[r]_{/M}} (\catC)} (A, B) \simeq \mathsf{Hom}_{\alg{\mfld[r]_{/M}} (\catC)} (\int A, \int B).
    \end{equation}
    %
    The existence of these is exactly the requirement that factorization homology in the usual sense is fully faithful, which is part of the statement of \cite[lem.2.17]{aft_fhstrat}.
\end{proof}

\begin{remark}
    One perspective on this result, provided by \cite[thm.2.43]{aft_fhstrat}, is that the theory of locally constant factorization algebras captures as much information as (generalized) homology theory in the sense of \cite[def.2.37]{aft_fhstrat}.
\end{remark}

\begin{remark}
    A version of \Cref{thm:disk_alg=lcfa} was proven as \cite[thm.6]{gtz2014} in the case of smooth manifolds. The proof there, however, made crucial use of choosing a Riemannian metric on the manifold and constructing geodesically convex neighborhoods. This technique is, at least presently, not extendable to the stratified case, but our proof above subverts the need for it. 
\end{remark}

\begin{remark}
    As noted at the beginning of \S2, the physical interpretation of this version of factorization homology is a way to take a spacetime $M$ and its local observables $A$, and construct \emph{some}, though typically not all, of the global observables. The previous \Cref{thm:disk_alg=lcfa} and \Cref{cor:disk_alg=falg} transfer this observation to factorization algebras, which aligns with the observation of \cite{cg2016} that the current version of factorization algebras can only deal with perturbative quantum field theories. 
\end{remark}

\begin{remark}
    The results of \Cref{thm:disk_alg=lcfa} and \Cref{cor:disk_alg=falg} can be seen as lowering the amount of data that is necessary to define a factorization algebra. There is a similar, but slightly weaker result in \cite[\S7.2]{cg2016}, which nonetheless works for all Hausdorff topological spaces. The key tool for them are factorizing bases, which play the role of Weiss covers but are further required to be closed under finite intersections. Their result then says that a factorization algebra defined on such a basis $\mathscr{U}$ can be extended to the full space $X$, namely there is an equivalence
    %
    \begin{equation}
        \begin{tikzcd}
            \falg_X(\catC) \arrow[rr, bend left=10, "\mathrm{-|_{\mathscr{U}}}"] \arrow[rr, phantom, "\simeq"]&  & \falg_{\mathscr{U}}(\catC) \arrow[ll, bend left=10, "\mathrm{ext}"].
        \end{tikzcd}
    \end{equation}
    %
    In comparison, our result drops the intersection requirement, while maintaining the statement.
\end{remark}

The $\infty$-operad $\disk_{/M}$ is, in general, harder to deal with than its symmetric monoidal counterpart $\disk(\bstr)$. The next lemma outlines a certain situation where this is not the case.

\begin{lemma}\label{lem:disk/M_to_disk}
    Let $M$ be a stratified manifold. If we can find an $\infty$-category of basics $\bstr_M$ such that $M$ admits a $\bstr_M$-structure and such that
    %
    \begin{equation}
        \mathsf{Hom}_{\mfld(\bstr_M)} (V, M) \simeq *
    \end{equation}
    %
    for all $V \in \bstr_M$, then there is an equivalence of symmetric monoidal $\infty$-categories
    %
    \begin{equation}
        \mathsf{Env}(\disk_{/M}) \xrightarrow{\ \ \simeq \ \ } \disk(\bstr_M).
    \end{equation}
\end{lemma}

\begin{remark}
    If we limit $M$ to be a basic then the requirements of the lemma are exactly that $M$ is final in $\bstr_M$.
\end{remark}

\begin{proof}
    We follow the proof of \cite[cor.2.33]{aft_fhstrat}. Using \Cref{rem:disk_b=disk_bsc}, the forgetful functor from the slice $\infty$-operad, which is a map of $\infty$-operads, is what will provide the equivalence
    %
    \begin{equation}
        \disk_{/M} \simeq \disk(\bstr_M)_{/M} \xrightarrow{\ \ \ \ } \disk(\bstr_M).
    \end{equation}
    %
    This functor is an equivalence on maximal $\infty$-subgroupoids because for any $V \in \disk(\bstr_M)$ the space of morphisms to $M$ is contractible, so, up to equivalence it is unique. Full faithfulness is provided by the fact that the functor on the active $\infty$-subcategories $\disk_{/M} \xrightarrow{} \disk(\bstr_M)$ is a right fibration. 
\end{proof}

The lemma says that for stratified manifolds $M$ that satisfy the conditions, the classification of locally constant factorization algebras on $M$, i.e. algebras over $\disk_{/M}$ simplifies to the classification of $\disk(\bstr_M)$-algebras in the usual sense, which are typically easier to work with,
%
\begin{equation}
    \lcfa_M (\catC) \simeq \mathsf{Fun}^\otimes (\mathsf{Env}(\disk_{/M}), \catC) \simeq \alg{\disk(\bstr_M)} (\catC).
\end{equation}

\subsection{Operations on Factorization Algebras}

There are a few important operations that we can do with factorization algebras (and the other mentioned variants) that will be important for our considerations. They allow us to compare factorization algebras from different spaces and will play a big role in any kind of classification statement one might make.


\begin{proposition}\label{prop:pushforward}
    Given a continuous map $f:X \rightarrow Y$ between topological spaces, there are pushforward functors
    %
    \begin{align}
        &f_*: \alg{\opens[s](X)} \xrightarrow{\quad} \alg{\opens[s](Y)} &f_*: \falg_X \xrightarrow{\quad} \falg_Y,
    \end{align}
    %
    which, on objects, are given by the prescription
    %
    \begin{equation}
        f_*F (U) := F(f^{-1}U).
    \end{equation}
\end{proposition}

\begin{proof}
    By the definition of continuity, the case of prefactorization algebras is trivial. Thus, the only thing we are left to show is that if $F$ satisfies the Weiss cosheaf property then so does $f_* F$. Given a Weiss cover $\{U_i \xhookrightarrow{} U\}_{i \in I}$ of $U$, we observe that $\{f^{-1}U_i \xhookrightarrow{} f^{-1}U\}_{i \in I}$ is a Weiss cover of $f^{-1}U$. Namely, if $S \subset f^{-1}U$ is a finite subset of $U$, then $f(S)$ is contained in some $U_i$ by the Weiss cover property of $\{U_i \xhookrightarrow{} U\}_{i \in I}$. But in that case $S$ has to be contained in $f^{-1}U_i$, giving it the Weiss cover property too. Thus, by the definition of the pushforward on objects the Weiss cosheaf property is preserved.
\end{proof}

The case of locally constant factorization algebras is different when it comes to the pushforward. This is because it's not immediate that local constancy is preserved when pushed forward. However, if we limit the types of maps we pushforward with we can still construct a functor. The following two results give the flavor of what is required.

\begin{proposition}[{\cite[prop.15]{ginot2015}}]
    If $f: X \rightarrow Y$ is a locally trivial fibration between smooth manifolds then the pushforward functor exists even between the $\infty$-categories of locally constant factorization algebras
    %
    \begin{equation}
        f_*: \lcfa_X \xrightarrow{\quad} \lcfa_Y.
    \end{equation}
\end{proposition}

\begin{corollary}\label{cor:pushforward_for_lc}
    Let $f: M \xrightarrow{} N$ be a constructible bundle of stratified manifolds. Then the pushforward functor for factorization homology of \Cref{thm:fh_pushforward} serves as a pushforward
    %
    \begin{equation}
        f_*: \lcfa_M \xrightarrow{\quad} \lcfa_N,
    \end{equation}
    %
    by using \Cref{thm:disk_alg=lcfa}.
\end{corollary}

\begin{remark}\label{rem:fh_is_pushingforward}
    Having defined the pushforward functor of factorization algebras we can consider the map $\mathsf{p}:M \rightarrow *$. Given any $A \in \alg{\disk(\bstr)_{/M}}$ the factorization algebra $F_A$ generated by $A$ satisfies
    %
    \begin{equation}
        \int_M A  = F_A(M) \simeq (\mathsf{p}_* F_A)(*).
    \end{equation}
    %
    The one point manifold is a very simple space, which makes the concepts of prefactorization algebras, factorization algebras and locally constant factorization algebras coincide. All of them are in fact given by a pointed object of the target category. Evaluation at $*$ simply returns the underlying object. Thus, in a sense evaluating factorization homology on a space $M$ is the same procedure as pushing-forward by the map $\mathsf{p}$. Versions of factorization homology that can evaluate lower dimensional manifolds (as compared to the structure of the disk algebra), for example as in \cite[cor.2.29]{aft_fhstrat}, can then be seen as relaxing this pushforward from $*$ to a manifold with more structure.
\end{remark} 

\begin{lemma}\label{lem:ff_functor_to_refinement}
    Let $u: X \rightarrow \hat{X}$ be the morphism of \Cref{ex:forget_strat} that forgets the stratification.
    The functor $u_*: \lcfa_{X} (\catC) \rightarrow \lcfa_{\hat{X}} (\catC)$ has a fully faithful left adjoint.
\end{lemma}

\begin{proof}
    Consider the functor $\hat{-}: \falg_{\hat{X}} \rightarrow \falg_{X}$, which acts as $\hat{F}(U) := F(u(U))$, namely it returns algebras that evaluate open subsets by forgetting their stratification first. We claim this functor is the left adjoint of $u_*$. Indeed, given a morphism $(\phi: \hat{F} \rightarrow G) \in \falg_{X}$, i.e. a family of morphisms $\phi(U): \hat{F}(U) \rightarrow G(U)$ for all opens $U$ in $X$, it is easy to verify that we can construct a morphism of algebras over $\hat{X}$, $\hat{\phi}(u(U)): F(u(U)) \rightarrow u_*G (u(U))$, since $\hat{F}(U) = F(u(U))$ and $u_*G(u(U)) = G(u^{-1}(u(U))) = G(U)$. The last equality holds because $u$ is an injective map. In fact, this last observation also proves that the unit of this adjunction is an isomorphism, granting the full faithfulness.
\end{proof}

\begin{remark}
    In fact, the above proof is true more generally for any map of stratified spaces that is a homeomorphism of the underlying topological spaces (so, in particular, any refinement). This means that locally constant factorization algebras record the data of coarser stratified structures as $\infty$-subcategories. Forgetting local constancy, the proof, of course, applies to factorization algebras of all flavors, however for the types that don't detect the stratified structure the above functor is even an equivalence, and not just fully-faithful, and is of no interest.
\end{remark}


In the opposite direction, we do not, in general, have a pullback functor. \cite{cg2016} provide a construction of a pullback in the case of open immersions, but we will only focus on the case of restrictions. Namely, given an open subset $U \subset X$ of a topological space $X$, and given a factorization algebra $F \in \falg_X$ we can clearly define a factorization algebra $F|_{U} \in \falg_U$ by restricting to those open subsets that are fully contained in $U$. Furthermore, this clearly holds not only for factorization algebras, but prefactorization algebras and locally constant factorization algebras too.

In the special case of product spaces $X \times Y$, we now know that the projections $\pi_1: X \times Y \rightarrow X$ and $\pi_2: X \times Y \rightarrow Y$, give us pushforwards. For $\pi_1$, for example, this essentially uses the fact that if we can evaluate all open sets of $X \times Y$ then we can definitely evaluate the open sets that look like $U \times Y$, with $U \subset X$ an open set. But in this case we can do even more, because we can even evaluate the more granular open subsets $U \times V$ with $U \subset X$ and $V \subset Y$ opens. More formally, we claim:
%
\begin{proposition}\label{prop:exp_of_products}
    Let $X$ and $Y$ be topological spaces. There is a functor from the $\infty$-category of prefactorization algebras on the product $X \times Y$ to the $\infty$-category of prefactorization algebras on $X$ valued in the category of prefactorization algebras on $Y$
    %
    \begin{equation}
        \bar{\pi}: \alg{\opens(X \times Y)}(\catC) \xrightarrow{\qquad} \alg{\opens(X)} (\alg{\opens(Y)} (\catC)).
    \end{equation}
    %
    This functor descends to factorization algebras too
    %
    \begin{equation}
        \bar{\pi}: \falg_{X \times Y}(\catC) \xrightarrow{\qquad} \falg_{X}(\falg_{Y}(\catC)).
    \end{equation}
\end{proposition}

\begin{remark}
    The conditions we imposed on $\catC$ allowed us, in \Cref{rem:sym_mon_inheritance}, to inherit a symmetric monoidal structure on the $\infty$-categories of factorization algebras of all varieties. It also allowed us to inherit the existence of colimits. This is the reason why the above can even be stated.
\end{remark}

\begin{proof}
    Indeed, given opens $U \subset X$ and $V \subset Y$, the values we assign are $(\bar{\pi}F) (U)(V) = F(U \times V)$. Fixing an open subset $U \subset X$, we obviously get a prefactorization algebra $(\pi_2|_U)_* (F|_U) \in \alg{\opens(Y)}$ by pushing forward with $\pi_2|_U: U \times Y \rightarrow Y$, which shows that $\bar{\pi}F$ is valued in $\alg{\opens(Y)}$. The only thing left to show is that $\bar{\pi} F$ has the structure of a prefactorization algebra on $X$. To do this, for each multi-morphism $(\{ U_i \}_{i \in I} \rightarrow U) \in \opens(X)$ we assign a map $\bar{\pi}F (\{ U_i \}_{i \in I}, U)$ of factorization algebras on $Y$. Such an algebra map is specified by giving its value on all opens of $Y$ separately. We can thus assign
    %
    \begin{equation}
        \bar{\pi}F (\{ U_i \}_{i \in I}, U)(V) = F(\{ U_i \times V \}_{i \in I}, U \times V),
    \end{equation}
    %
    which uses the projections $\pi_1|_V : X \times V \rightarrow X$ to assign $\pi_1|_V^{-1}(U) = U \times V$.

    It is not hard to check that the above also holds for the example of factorization algebras on topological spaces, by using \cite[\S7.2]{cg2016}. However, we'll focus on manifolds, in which case, the construction from above immediately works since we now have to do it for $\disk[r]_{/X}$ instead of $\mfld[r]_{/X}$. The only things to mention is that products of disks are again disks \cite[cor.3.4.9]{aft_localstrut}.
\end{proof}


\begin{proposition}\label{prop:exp_of_products_lc}
    Let $M$ and $N$ be stratified manifolds. The functor $\bar{\pi}$ descends to locally constant factorization algebras, and moreover, it is an equivalence of $\infty$-categories
    %
    \begin{equation}
        \bar{\pi}: \lcfa_{M \times N}(\catC) \xrightarrow{\ \ \simeq \ \ } \lcfa_{M}(\lcfa_{N}(\catC)).
    \end{equation}
\end{proposition}

\begin{remark}
    The above proposition also appears as \cite[prop.18]{ginot2015}. The equivalence proven there is for the case of smooth manifolds and relies on the local proof of \cite{lurie_ha} for $\mathbb{R}^n$. Using \Cref{thm:disk_alg=lcfa}, the statement is implied by the fact that there is an equivalence of $\infty$-operads
    %
    \begin{equation}
        \disk_{/M} \otimes \disk_{/N} \xrightarrow{\ \ \simeq \ \ } \disk_{/M \times N},
    \end{equation}
    %
    with the tensor product of $\infty$-opeards. Again for smooth manifolds this is found in \cite[ex.5.4.5.5]{lurie_ha}.
\end{remark}

\begin{proof}
    The proof of existence of the functor goes along the same lines as the proof of \Cref{prop:exp_of_products} because of \Cref{thm:disk_alg=lcfa}. The thing to take into account is that the projection maps are both, trivially, constructible bundles so by \Cref{cor:pushforward_for_lc}, pushing forward by them yields locally constant factorization algebras again. This combined with the fact that for any $\infty$-operads $\mathscr{O}$, $\mathscr{O}'$ and $\mathscr{C}$ we have an equivalence
    %
    \begin{equation}
        \alg{\mathscr{O}} (\alg{\mathscr{O}'} (\catC)) \simeq \alg{\mathscr{O}'} (\alg{\mathscr{O}} (\catC))
    \end{equation}
    %
    allows us to conclude that the functor lands in locally constant factorization algebras
    %
    \begin{equation}
        \bar{\pi}: \lcfa_{X \times Y}(\catC) \xrightarrow{\ \ \ \ } \lcfa_{X}(\lcfa_{Y}(\catC)).
    \end{equation}

    One way to show the equivalence is to use factorization homology, as in \Cref{thm:disk_alg=lcfa}, to write a locally constant factorization algebra as $\int_{-} A \in \lcfa_{X \times Y} (\catC)$. Then using the Fubini theorem \cite[cor.2.29]{aft_fhstrat}\footnote{The notation used in the citation is technically abusive because factorization homology is evaluating lower dimensional manifolds, but it is clear that this is to be understood exactly as the evaluation of the pushforward factorization algebra on the lower dimensional manifold (see \Cref{rem:fh_is_pushingforward}).} for factorization homology we get essential surjectivity because there are equivalences
    %
    \begin{equation}
        \int_{U \times V} A \xrightarrow{\ \ \simeq \ \ } \int_U \int_V \bar{\pi} A
    \end{equation}
    %
    between objects of $\catC$. As argued in the proof of \Cref{thm:disk_alg=lcfa}, these equivalences are natural in the open set variables, which means that they rise to equivalences at the level of locally constant factorization algebras. In fact, looking at the proof of \cite[thm.2.25]{aft_fhstrat}, which is what underlies the construction of the above equivalences, we see that they arise from an argument about a final functor. By \Cref{rem:fin_equiv_natural} this means that the equivalences are actually natural in the algebra variable, so it is immediate that the equivalences of individual locally constant factorization algebras lift to a full equivalence of $\infty$-categories.
\end{proof}




\subsection{Factorization Algebras on Intervals}\label{ssec:lcfa_on_ints}

As was the case with factorization homology, so too with factorization algebras we now turn to a few very important constructions of locally constant factorization algebras on intervals. Namely, we will consider how we can construct a locally constant factorization algebra on the spaces $\mathbb{R}$ (a.k.a. the open interval) and $\hoint$ (a.k.a. the half-open interval). These will be a very important backbone for intuition about locally constant factorization algebras, as well as a key part of proofs later on. We finish with factorization algebras for the closed interval $[-1,1]$.

\begin{construction}\label{con:lcfas_on_R}
    Consider a unital, (homotopy) associative algebra $A$ in the category $\catC$. For a first pass we use \cite[\S7.2]{cg2016} to construct a factorization algebra out of this data. We first check that the disks $\disk[r]_{/\mathbb{R}}$ of $\mathbb{R}$ form a factorizing basis. This is true since the disks are always a basis for any manifold, they are factorizing since we have all disjoint unions, and they are closed under finite intersections because the disks of $\mathbb{R}$ are intervals. Therefore, by \cite[\S7.2]{cg2016}, we need only assign values to these. Given any disk $U$ we assign the same value, $F_A(U) := A$. Similarly, given a multi-morphism $\sqcup_{i \in I} U_i \xhookrightarrow{} U$, with $I$ non-empty, we assign the map given by the multiplication of $A$
    %
    \begin{equation}
        F_A(\sqcup_{i \in I} U_i \xhookrightarrow{} U) := \otimes_{i \in I} A \rightarrow A.
    \end{equation}
    %
    The implicit convention, is that the ordering of the intervals, provided by the standard orientation (or framing) of $\mathbb{R}$, determines the order in which we multiply. Associativity forbids any other ordering except the completely opposite one, which would yield the opposite multiplication, and thus the opposite algebra. Finally, in the case of $\emptyset \xhookrightarrow{} U$, we assign the map that chooses the unit $\mathbb{1} \rightarrow A$. At this point we have constructed a prefactorization algebra on $\mathbb{R}$. To use \cite[\S7.2]{cg2016} we need to first check the Weiss cosheaf condition. But because intervals are closed under finite intersections the \v{C}ech nerve definition of \Cref{rem:check_nerve} makes this easy; the simplicial object $\cech_{\bullet}(\disk[r]_{/\mathbb{R}}, F_A)$ in all degrees will be a coproduct of copies of $A$. Thus, $F_A$ is indeed a $\disk[r]_{/\mathbb{R}}$-factorization algebra and can be extended to a factorization algebra on $\mathbb{R}$. By definition, it is also obviously locally constant.

    In this simple case the disks are closed under finite intersections, so the construction works out. More generally, one would need to find another factorizing basis. In the case of smooth manifolds one possible way is to choose a Riemannian metric and take the geodesically convex neighborhoods. An alternative is to get rid of the property of being closed under finite intersections altogether. It turns out, because of \Cref{thm:disk_alg=lcfa}, that this is indeed possible, and that we can always stick to using disks even in the most general situations.

    In whatever way we go about it, it is immediate that we have constructed a functor $\alg{\mathbb{E}_1}(\catC) \rightarrow \lcfa_{\mathbb{R}} (\catC)$ from the $\infty$-category of unital, (homotopy) associative algebras, or $\mathbb{E}_1$-algebras to the $\infty$-category of locally constant factorization algebras on $\mathbb{R}$. In fact, we claim that this functor is actually an equivalence, i.e., that every locally constant factorization algebra on $\mathbb{R}$ is equivalent to one induced like in the construction.
\end{construction}

\begin{proposition}\label{prop:R_gives_E1}
    There is an equivalence of $\infty$-categories
    %
    \begin{equation}
        \alg{\mathbb{E}_1}(\catC) \xrightarrow{\ \ \simeq \ \ } \lcfa_{\mathbb{R}} (\catC).
    \end{equation}
\end{proposition}

\begin{proof}
    What is left to show is the essential surjectivity of the functor. This holds because every disk in $\mathbb{R}$ is diffeomorphic to $\mathbb{R}$ itself. So, given $F \in \lcfa_{\mathbb{R}}$, we can get an underlying object for our $\mathbb{E}_1$-algebra as $A = F(\mathbb{R})$. The previous observation and the local constancy then force the value on each disk to be equivalent to this. The space of embeddings of two disks $\mathbb{R} \sqcup \mathbb{R} \xhookrightarrow{} \mathbb{R}$ is, up to homotopy, equivalent to two points, which provide exactly the multiplication map $ A \otimes A \rightarrow A$ of the algebra and its opposite multiplication.  Associativity, up to homotopy, comes about exactly because all embeddings of any number of disks can, up to homotopy, be factorized through the embedding of two disks at a time. As before, the unit comes from the map $F(\emptyset \xhookrightarrow{} \mathbb{R}) = (\mathbb{1} \rightarrow A)$. Thus, we have constructed an $\mathbb{E}_1$-algebra from the given data, and the proposition holds.
\end{proof}

\begin{construction}\label{con:lcfas_on_hoint}
    Consider a unital, (homotopy) associative algebra $A$ in the category $\catC$, as above, together with a unital right module $M$ of $A$ in the category $\catC$. From this data we will construct a locally constant factorization algebra $F_{(M, A)}$ on the stratified space $\hoint$. Considering \Cref{con:lcfas_on_R} and the open embedding $\mathbb{R} \cong \mathbb{R}_{>0} \xhookrightarrow{} \hoint$, we will choose the restriction to be $F_{(M,A)}|_{\mathbb{R}} = F_A$, i.e. any disk that doesn't include the point $0 \in \hoint$, we regard as a disk of $\mathbb{R}$ through the given map, and we assign the value $A$ to it like in the \Cref{con:lcfas_on_R}. This leaves us only with the stratified disks $U_*$ that contain the point $0 \in \hoint$. To these we assign the module $F_{(M,A)}(U_*) = M$. Since there can be at most one stratified disk that includes the point $0 \in \hoint$, the maps that we have to assign values to look like $U_* \sqcup (\sqcup_{i \in I} U_i) \xhookrightarrow{} U'_*$, but this is exactly what is provided by the right module structure
    %
    \begin{equation}
        F_{(M,A)} (U_* \sqcup (\sqcup_{i \in I} U_i) \xhookrightarrow{} U'_*) = M \otimes (\otimes_{i \in I} A) \rightarrow M.
    \end{equation}
    %
    Of course the pointing of the module comes from $F_{(M,A)}(\emptyset \xhookrightarrow{} U_*) = \mathbb{1} \rightarrow M$. As before, even more is true:
\end{construction}

\begin{proposition}\label{prop:hoint_gives_modules}
    There is an equivalence of $\infty$-categories between the $\infty$-categories of $\mathbb{E}_1$-algebras together with a right module and locally constant factorization algebras on the half-open interval
    %
    \begin{equation}
        \mathsf{RMod}(\catC) \xrightarrow{\ \ \simeq \ \ } \lcfa_{\hoint} (\catC).
    \end{equation}
\end{proposition}

\begin{proof}
    The proof goes along the same lines as the proof of \Cref{prop:R_gives_E1}. We can find the module by evaluating on $\hoint$, and we can find the algebra by evaluating on $\hoint \setminus \{ 0 \} \cong \mathbb{R}$, as a specified stratum of $\hoint$.
\end{proof}

\begin{remark}\label{rem:left_or_right_module}
    The fact that we chose left modules to provide the data in \Cref{prop:hoint_gives_modules}, does not play much of a role because a left module is exactly a right module of the opposite algebra, giving $\mathsf{LMod}(\catC) \simeq \mathsf{RMod}(\catC)$. Where the distinction comes up, is if we want to fix the $\mathbb{E}_1$-algebra away from $0 \in \hoint$ beforehand. In that case we have to use one of the two possible conventions on the order of multiplication. If we choose the increasing convention then the subset containing $0 \in \hoint$ is on the left making it a right module, while with the opposite convention it would be a left module of the opposite algebra.
\end{remark}

It's not hard to extend \Cref{con:lcfas_on_R} and \Cref{con:lcfas_on_hoint} to the closed interval $[-1,1]$ by hand, but through \Cref{thm:disk_alg=lcfa} we can see that \Cref{prop:disk1bor_over_int=ORL} and \Cref{prop:env_ORL=env_assocRL} already do all the necessary work.

\begin{corollary}\label{cor:lcfas_on_closed_int}
    There is an equivalence between the $\infty$-categories of locally constant factorization algebras on the closed interval and algebras over the $\infty$-operad $\mathsf{Assoc^{RL}}$
    %
    \begin{equation}
        \lcfa_{[-1,1]} (\catC) \xrightarrow{\ \ \simeq \ \ } \alg{\mathsf{Assoc^{RL}}} (\catC).
    \end{equation} 
\end{corollary}

\begin{remark}
    Another perspective on why all of these constructions look the way they do is provided by \Cref{lem:disk/M_to_disk}. It applies in all of the above cases, namely:
    %
    \begin{enumerate}
        \item For $\mathbb{R}$, the convenient $\infty$-category of basics is $\bstr = \mathsf{D}_1^*$,
        \item For $[-1,1]$ the convenient $\infty$-category of basics is $\bstr = \mathsf{D}_1^{\partial, *}$.
        \item For $\hoint$, we could define a convenient $\infty$-category of basics which describes framed 1-dimensional manifolds which only have a right boundary (a concept that we can state because of the framing).
    \end{enumerate}
    %
    We can see that even though factorization algebras, as we have defined them here, don't detect the tangential structure of the manifold, it is usually favorable, if possible, to give the manifold some structure anyway. In particular, the rigid structure of framing is very useful because it massively reduces the spaces of allowed open embeddings.
\end{remark}




\subsection{Factorization Algebras on Euclidean Spaces}\label{ssec:lcfa_on_Rn}

In this section we want to describe locally constant factorization algebras on Euclidean spaces $\mathbb{R}^n$. By \Cref{lem:disk/M_to_disk} we can hope for an $\infty$-category of basics such that we simplify the situation down to $\disk (\bstr)$-algebras. In fact, since $\mathbb{R}^n$ is frameable an $\infty$-category of basics that we can choose is $\mathsf{D}_n^* \simeq * \xrightarrow{\{ \mathbb{R}^n\}} \bsc$. Specifically, this makes us focus on framed $n$-dimensional disk algebras
%
\begin{equation}
    \lcfa_{\mathbb{R}^n} (\catC) \simeq \alg{\disk (\mathsf{D}_n^*)} (\catC).
\end{equation}
%

\begin{theorem}[{\cite[thm.5.5.4.10]{lurie_ha}}]
    There is an equivalence of $\infty$-categories between the $\infty$-categories of locally constant factorization algebras on $\mathbb{R}^n$ and $\mathbb{E}_n$-algebras
    %
    \begin{equation}
        \alg{\mathbb{E}_n} (\catC) \xrightarrow{\ \ \simeq \ \ } \lcfa_{\mathbb{R}^n} (\catC).
    \end{equation}
\end{theorem}

\begin{proof}
    With our setup, thanks to \Cref{thm:disk_alg=lcfa} and \Cref{lem:disk/M_to_disk}, it is sufficient to show that there is a weak equivalence of $\infty$-operads between $\disk (\mathsf{D}_n^*)$ and $\mathbb{E}_n$, i.e.
    %
    \begin{equation}
        \alg{\disk (\mathsf{D}_n^*)} \xrightarrow{\ \ \simeq \ \ } \alg{\mathbb{E}_n}.
    \end{equation}
    %
    But this is just the fact that the $\disk_{/M}$ $\infty$-operad gives rise to $\mathbb{E}_n$-algebras from \cref{prop:framed_ndisk=En}.
\end{proof}

\begin{remark}
    We already encountered the $n=1$ version of the proposition above when we discussed locally constant factorization algebras on the open interval in \Cref{prop:R_gives_E1}, and the construction there gives us an intuition about how to actually construct the locally constant factorization algebra from the data of the $\mathbb{E}_n$-algebra. The theorem formalizes that this assignment is, in fact, an equivalence.
\end{remark}

\begin{remark}[{\cite[prop.3.16]{francis2013}, \cite[ex.5.5.4.16]{lurie_ha}}]
    In the case of Euclidean spaces like $\mathbb{R}^n$, essentially because of translation invariance, a special thing happens when we restrict away from the origin. That is, we know that given a locally constant factorization algebra $F$ on $\mathbb{R}^n$ we can restrict it to any open subset like, for example, $\mathbb{R}^n \setminus \{ 0 \} \cong S^{n-1} \times \mathbb{R}$ to get another locally constant factorization algebra $F|_{S^{n-1} \times \mathbb{R}}$. By \Cref{prop:exp_of_products_lc} we know that $F|_{S^{n-1} \times \mathbb{R}} \in \lcfa_{\mathbb{R}} (\lcfa_{S^{n-1}} (\catC))$, so that in particular it is an $\mathbb{E}_1$-algebra. If we pushforward to $\mathbb{R}$, by the discussion at \Cref{rem:fh_is_pushingforward} we could denote the resulting $\mathbb{E}_1$-algebra by
    %
    \begin{equation}
        \int_{S^{n-1}} F.
    \end{equation}
    %
    What \cite[prop.3.16]{francis2013} shows is then:

    \begin{proposition}\label{univ_env_for_En}
        Given a locally constant factorization algebra $F$ on $\mathbb{R}^n$, its universal enveloping algebra is given by $\int_{S^{n-1}} F$, i.e. there is an equivalence of $\infty$-categories
        %
        \begin{equation}
            \mathsf{Mod}^{\mathbb{E}_n}_{F} (\catC) \xrightarrow{\ \ \simeq \ \ } \mathsf{LMod}_{\int_{S^{n-1}} F} (\catC).
        \end{equation}
    \end{proposition}
    %
    This result goes towards an explanation of the appearance of factorization homology over higher spheres in the classification statement of \Cref{thm:classif_disk_d_under_n*_alg}.
\end{remark}

We can view a Euclidean space $\mathbb{R}^{m+n}$ as a product space $\mathbb{R}^m \times \mathbb{R}^n$, in which case \Cref{prop:exp_of_products_lc} specializes to give the famous result of Dunn additivity \cite{dunn1988} which can be found in the following form in \cite{lurie_ha}:

\begin{proposition}\label{prop:dunn_additivity}
    There is an equivalence of $\infty$-categories
    %
    \begin{equation}
        \alg{\mathbb{E}_{n + m}} (\catC) \xrightarrow{\ \ \simeq \ \ } \alg{\mathbb{E}_n} (\alg{\mathbb{E}_m} (\catC)). 
    \end{equation}
\end{proposition}

\begin{remark}
    The proof of \Cref{prop:exp_of_products_lc}, when specialized to ordinary manifolds, as given in \cite{ginot2015} actually depends on Dunn additivity to work, however the proof presented above and the one in \cite{lurie_ha} are independent. 
\end{remark}


\end{document}