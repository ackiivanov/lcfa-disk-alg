\documentclass[../text]{subfiles} 

\begin{document}


\section{\texorpdfstring{$\mathbb{E}_n$}{En}-algebras}\label{app:En_algebras}

The operad governing $\mathbb{E}_n$-algebras was first defined in \cite{bv73} as a topological operad, and has since become ubiquitous in algebraic topology. As we saw in \cref{prop:framed_ndisk=En} one incarnation of this operad is as framed embeddings of disks into other disks. Here we follow the exposition of \cite[sec.5.1]{lurie_ha} to define the operad differently, and see how it can be given the structure of an $\infty$-operad.

\begin{definition}\label{def:rect_embedding}
    Let $n$ be an integer, and $S$ be a finite set. We will call a map $f:(-1,1)^n \times S \rightarrow (-1,1)^n$ a \emph{rectilinear embedding} if it is given by
    %
    \begin{equation}
        f(x_1, \dots, x_n, s) = (a^s_1 x_1 + b^s_1, \dots, a^s_n x_n + b^s_n),
    \end{equation}
    %
    where $a^s_i, b^s_i \in \mathbb{R}$ and $a^s_i > 0$ for all $s \in S$ and all $1 \leq i \leq n$. We denote the set of such rectilinear embeddings as $\mathsf{Rect}((-1,1)^n \times S \rightarrow (-1,1)^n)$.
\end{definition}

\begin{remark}
    The set $\mathsf{Rect}((-1,1)^n \times S \rightarrow (-1,1)^n)$ has a natural topology on it given by the subspace topology, when considered as an open subset of $\mathbb{R}^{2n |S|}$, by the map
    %
    \begin{equation}
        \mathsf{Rect}((-1,1)^n \times S \rightarrow (-1,1)^n) \xrightarrow{\quad} \mathbb{R}^{2n |S|},
    \end{equation}
    %
    which sends a rectilinear embedding to the tuple of its coefficients as in \cref{def:rect_embedding}.
\end{remark}

Classically, the $\mathbb{E}_n$ operad was defined as a topological operad according to the following:

\begin{definition}
    Let $n$ be an integer. As a topological operad, $\mathrm{E}_n$ is defined by the collection of spaces
    %
    \begin{equation}
        \left\{ \mathsf{Rect}((-1,1)^n \times \langle k \rangle^\circ \rightarrow (-1,1)^n)\right\}_{k \geq 0},
    \end{equation}
    %
    where $\langle k \rangle^\circ = \{1, 2, \dots, k\}$ and where:
    %
    \begin{enumerate}
        \item The symmetric group $\Sigma_k$ acts on $\mathsf{Rect}((-1,1)^n \times \langle k \rangle^\circ \rightarrow (-1,1)^n)$ by permutations in the finite set $\langle k \rangle^\circ$.
        \item The composition maps are given induced by the maps
        %
        \begin{equation}
            \langle k_1 \rangle^\circ \times \dots \langle k_m \rangle^\circ \xrightarrow{\quad} \langle k_1 + \dots + k_m \rangle^\circ.
        \end{equation}
        \item The unit is given by the map $\mathrm{id} : (-1,1)^n \times \langle 1 \rangle^\circ \cong (-1,1)^n \rightarrow (-1,1)^n$.
    \end{enumerate}
\end{definition}

\begin{remark}
    This is why the operad is sometimes known as the operad of little $n$-cubes.
\end{remark}

Towards an $\infty$-operadic definition, we can construct, now a little more categorically, a topological category $\mathbb{E}_n$.

\begin{definition}[{\cite[def.5.1.0.2]{lurie_ha}}]
    Let $n$ be an integer. Define a topological category $\mathbb{E}_n$ with the following:
    %
    \begin{enumerate}
        \item The objects of $\mathbb{E}_n$ are the sets $\langle k \rangle = \{ *, 1, 2, \dots, k\} \in \mathsf{Fin}_*$.
        \item A morphism from $\langle k \rangle$ to $\langle l \rangle$ is given by a morphism $(\alpha : \langle k \rangle \rightarrow \langle l \rangle) \in \mathsf{Fin}_*$ together with a rectilinear embedding $(-1,1)^n \times \alpha^{-1}\{j\} \xrightarrow{\quad} (-1,1)^n$ for each $j \in \langle l \rangle^\circ$, such that the topology of the morphism space is given by 
        %
        \begin{equation}
            \mathsf{Hom}_{\mathbb{E}_n}(\langle k \rangle \langle l \rangle) = \coprod_{\alpha : \langle k \rangle \rightarrow \langle l \rangle} \prod_{1 \leq j \leq k} \mathsf{Rect} ((-1,1)^n \times \alpha^{-1}\{ j\}, (-1,1)^n).
        \end{equation}
        \item Composition of morphisms is, as usual, defined by partial concatenation.
    \end{enumerate}
\end{definition}

\begin{proposition}[{\cite[prop.5.1.0.3]{lurie_ha}, \cite[cor.1.1.5.12]{lurie_htt}}]
    The obvious functor between (the nerves) $\mathbb{E}_n \rightarrow \mathsf{Fin}_*$ exhibits $\mathbb{E}_n$ as an $\infty$-operad.
\end{proposition}

More intuitively, we can think of the $\mathbb{E}_n$ $\infty$-operad as defining homotopy associative algebras with various levels of commutativity:
%
\begin{enumerate}
    \item $\mathbb{E}_1$ is the operad that describes unital, homotopy associative algebras.
    \item $\mathbb{E}_2$ is the operad that describes unital, homotopy associative algebras for which there exist homotopies that make the multiplication commutative, but the space of those homotopies is generically not contractible.
    \item The higher $n$ is the more and more higher homotopies exist that make the multiplication and all its previous coherences commutative.
    \item In the limit $n \rightarrow \infty$, we get the $\infty$-operad $\mathbb{E}_{\infty}$ that governs homotopy commutative algebras with fully coherent homotopies
\end{enumerate}

A more detailed explanation of this intuitive picture can be found in \cite{tanaka20}.

\begin{remark}
    We should also remember that having defined things operadically, means that we have the freedom to place our algebra in a wide variety of $\infty$-operads or symmetric monoidal $\infty$-categories. However, if the target is the nerve of an ordinary category then it will not be able to tell the difference between $\mathbb{E}_n$-algebras, with $n \geq 2$, they will all look fully commutative.
\end{remark}


\end{document}