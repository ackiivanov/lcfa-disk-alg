\documentclass[]{amsart}

% from preamble ->

\usepackage[ngerman,french,british]{babel} % clashes with q.uiver/tikzcd
\usepackage[utf8]{inputenc}
\usepackage[T1]{fontenc}
\usepackage{amsthm}
\usepackage{amssymb}
\usepackage{mathtools} % \coloneqq
\usepackage{amsmath}
\usepackage[outline]{contour}
\usepackage[mathscr]{euscript}
\usepackage[all,2cell,cmtip]{xy}
\usepackage{csquotes}
\usepackage[usenames,dvipsnames]{xcolor}
\usepackage[
  %hidelinks
]{hyperref}
\usepackage{microtype}


% quiver ->

% `tikz-cd` is necessary to draw commutative diagrams.
\usepackage{tikz-cd}
% `amssymb` is necessary for `\lrcorner` and `\ulcorner`.
\usepackage{amssymb}
% `calc` is necessary to draw curved arrows.
\usetikzlibrary{calc}
% `pathmorphing` is necessary to draw squiggly arrows.
\usetikzlibrary{decorations.pathmorphing}

\usetikzlibrary{babel} % ADDED MYSELF, OTHERWISE CLASHES WITH TIKZ-CD KEYS -Ö

% A TikZ style for curved arrows of a fixed height, due to AndréC.
\tikzset{curve/.style={settings={#1},to path={(\tikztostart)
    .. controls ($(\tikztostart)!\pv{pos}!(\tikztotarget)!\pv{height}!270:(\tikztotarget)$)
    and ($(\tikztostart)!1-\pv{pos}!(\tikztotarget)!\pv{height}!270:(\tikztotarget)$)
    .. (\tikztotarget)\tikztonodes}},
    settings/.code={\tikzset{quiver/.cd,#1}
        \def\pv##1{\pgfkeysvalueof{/tikz/quiver/##1}}},
    quiver/.cd,pos/.initial=0.35,height/.initial=0}

% TikZ arrowhead/tail styles.
\tikzset{tail reversed/.code={\pgfsetarrowsstart{tikzcd to}}}
\tikzset{2tail/.code={\pgfsetarrowsstart{Implies[reversed]}}}
\tikzset{2tail reversed/.code={\pgfsetarrowsstart{Implies}}}
% TikZ arrow styles.
\tikzset{no body/.style={/tikz/dash pattern=on 0 off 1mm}}

% <- quiver

\usepackage[hyperref,
maxbibnames=99,
backend=biber,
%backref, %& "cit. on p.X",
%backrefstyle=none,
%style=numeric,
style=alphabetic-verb,
%citestyle=numeric,
citestyle=alphabetic-verb,
giveninits=true
%sorting=none
%citestyle=alphabetic
]{biblatex}
\DefineBibliographyStrings{english}{% %REMOVE "p." and "pp."
  page             = {},
  pages            = {},
} 
\renewbibmacro{in:}{} % gets rid of "In:[journal]"
\emergencystretch=1.5em
\addbibresource{text.bib}
\renewcommand*{\bibfont}{\footnotesize}





\usepackage{graphicx}
\usepackage{cleveref}   % use with \cref and \Cref
\usepackage{underscore} % use _ in text



%%%%%%%%%%%INCLUDING INKSCAPE %%%%%%%%%%%%%%
\usepackage{import}
\usepackage{color} % NECESSARY for inkscape for some reason
\usepackage{xifthen}

\usepackage{rotating} 
\newcommand{\incfig}[1]{%
    \def\svgwidth{0.3\columnwidth}
    \import{./figures/}{#1.pdf_tex}
}

\newcommand{\mcal}[1]{\mathcal{#1}}
\newcommand{\msc}[1]{\mathscr{#1}}
\newcommand{\msf}[1]{\mathsf{#1}}
\newcommand{\mscc}[1]{\mathbf{#1}}
\newcommand{\mrm}[1]{\mathrm{#1}}
\newcommand{\mbf}[1]{\mathbf{#1}}
\newcommand{\mbb}[1]{\mathbb{#1}}
\newcommand{\mfrak}[1]{\mathfrak{#1}}
\newcommand{\down}[3][ ]{(#2 \downarrow #3)^{#1}}
\newcommand{\paren}[1]{\left( #1 \right)}


\newcommand{\R}{\mbb{R}}
\newcommand{\Imm}{\mathrm{Im}}
\newcommand{\id}{{\mathrm{id}}}
\newcommand{\colim}{\mathrm{colim}}
\newcommand{\Hom}{\mathrm{Hom}}
\newcommand{\dfnt}{\bfseries \itshape}
\newcommand{\hyph}{{\mbox{-}}}
\newcommand{\Bred}{\mathcal{B}\mbox{-}\mathrm{red}}
\newcommand{\Yred}{Y\mbox{-}\mathrm{red}}
\newcommand{\Bopen}{\mathcal{B}\mbox{-}\mathrm{open}}
\newcommand{\Snglr}{\mathcal{S}\mathrm{nglr}}
\newcommand{\bighrule}{\vspace{\baselineskip} \hrule \hrule \hrule\vspace{\baselineskip}}
\newcommand{\solidone}{\mathbf{s}\ast_1} 
\newcommand{\Vinj}{{\mscc{V}^{\mathrm{inj}}}}
\newcommand{\Vhook}{\mscc{V}^{\hookrightarrow}}
\newcommand{\Vsim}{{\mscc{V}^{\simeq}}}
\newcommand{\Vlhook}{\mscc{V}^{\hookleftarrow}}
\newcommand{\TB}{T^{\mathcal{B}}}
\newcommand{\EX}{{\mathbf{Ex}}}
\newcommand{\Disk}{{\mathcal{D}\mathrm{isk}}}
\newcommand{\scB}{{\mathcal{B}}}
\newcommand{\OB}{{\mathbb{O}^{\mathcal{B}}}}
\newcommand{\Mnf}{{\mscc{M}\mathrm{nf}}}
\newcommand{\sfr}{{\mathbf{s}\ast}}
\newcommand{\solid}{\mathbf{s}}
\newcommand{\Bord}{{\mathrm{Bord}}}
\newcommand{\BM}{{\mathcal{B}\mathbf{M}}}
\newcommand{\pt}{{\mathrm{pt}}}
\newcommand{\falc}{{\mathrm{FA}^{\mathrm{lc}}}}
\newcommand{\stfr}{{\mathbf{st}\ast}}
\newcommand{\Tang}{{\mathcal{T}\mathrm{ang}}}
\newcommand{\opp}{{\mathrm{op}}}
\newcommand{\Opp}{{\mathrm{Op}}}
\newcommand{\Vsur}{{\mathcal{V}^{\twoheadrightarrow}}}
\newcommand{\EN}{{\mathbf{En}}}
\newcommand{\codim}{{\mathrm{codim}\,}}
\newcommand{\red}{{\text{-}\mathrm{red}}}
\newcommand{\Vect}{{\mathcal{V}}}
\newcommand{\Cinfty}{\mscc{C}\mathrm{at}_{\infty}}
\newcommand{\Sp}{{\mathcal{S}\mathrm{p}}}
\newcommand{\BhookO}{{\mathcal{B}^{\hookrightarrow}\mathrm{O}}}
\newcommand{\EhookO}{{\mathcal{E}^{\hookrightarrow}\mathrm{O}}}
\newcommand{\purp}{\color{purple}}
\newcommand{\Eop}{{\mathcal{E}^{\mathrm{op}}}}
\newcommand{\E}{{\mathcal{E}}}
\newcommand{\Tw}{{\mathrm{Tw}}}
\newcommand{\Map}{{\mathrm{Map}}}
\newcommand{\gpdinf}{{\mathcal{G}\mathrm{pd}_{\infty}}}
\newcommand{\cdon}{\[\begin{tikzcd}}
\newcommand{\cdoff}{\end{tikzcd}\]}
\newcommand{\ncdon}{\begin{equation}\begin{tikzcd}}
\newcommand{\ncdoff}{\end{tikzcd}\end{equation}}
\newcommand{\OO}{{\mathrm{O}}}
\newcommand{\pbarrow}{\ar[dr,phantom,"\ulcorner" very near start]}
\newcommand{\Gr}{{\mathrm{Gr}}}
\newcommand{\arrowedruleron}{{\vspace{5pt}{\purp\hrule\hrule\hrule}
\begin{center}\purp
	$\downarrow$
\end{center}}}
\newcommand{\arrowedruleroff}{{\begin{center}\purp
	$\uparrow$
\end{center}
\vspace{5pt}{\purp\hrule\hrule\hrule}}}
\newcommand{\blue}{\color{blue}}
\newcommand{\Sing}{{\mathrm{Sing}}}
\newcommand{\Bbox}{\mathcal{B}^{\oplus}}
\newcommand{\bbox}{B^{\oplus}}
\newcommand{\bboxop}{B^{\oplus}_{\opp}}
\newcommand{\Nerve}{{\mathrm{N}}}
\newcommand{\Ndelta}{\mathrm{N}^{\mbf{\Delta}}}
\newcommand{\Nhc}{\mathrm{N}^{\mathrm{hc}}}
\newcommand{\sSet}{\mathrm{sSet}}
\newcommand{\Cdelta}{{\mathcal{C}\mathrm{at}_{\mbf{\Delta}}}}
\newcommand{\pSh}{\mathrm{pSh}}
\newcommand{\EEx}{\mathbf{EX}}
\newcommand{\Pnsim}{\mscc{P}^{\nsim}}
\newcommand{\Phat}{\mscc{P}^{\Delta}}
\newcommand{\Pbox}{\mscc{P}^{\Box}}
\newcommand{\Nhat}{\widehat{\mathcal{N}}}
\newcommand{\Fun}{\mathrm{Fun}}
\newcommand{\Path}{\mathrm{Path}}
\newcommand{\Tangent}{\mathrm{T}}
\newcommand{\Normal}{\mathrm{N}}
\newcommand{\vfr}{\mathrm{vfr}}
\newcommand{\hty}{\mathrm{h}}
\newcommand{\evv}{\mathrm{ev}}
\newcommand{\Ar}{\mathrm{Ar}}
\newcommand{\inc}{$\infty$-category}
\newcommand{\btu}{\bigtriangleup}
\makeatletter
\newcommand{\btd}{\mathord{\mathpalette\raise@half\bigtriangledown}}
\newcommand\raise@half[2]{%
  \raisebox{.9\depth}{$\m@th#1#2$}%
}
\makeatother
\newcommand{\ashrk}{\text{\textexclamdown}}
\newcommand{\dash}{\mbox{-}}
\newcommand{\cat}[2]{\mscc{#1}\mathrm{#2}}
\newcommand{\htimes}{\times^{\mathrm{h}}}

\newcommand{\spans}{\mscc{S}^{\mrm{sp}}}
\newcommand{\OEEx}{\overrightarrow{\EEx}}
\newcommand{\Ssp}{\mscc{S}^{\mrm{sp}}}
\newcommand{\ocat}[1]{\mathrm{#1}}
\newcommand{\Ctop}{\mscc{C}\mrm{at}_{\mrm{Top}}}
\newcommand{\Tame}{\Hom^{t}}
\newcommand{\tame}{\tau}
\newcommand{\Unzip}{\mrm{Unzip}}

\numberwithin{equation}{section}

\theoremstyle{definition}
\newtheorem{definition}[equation]{Definition}%[section]
\newtheorem*{definition*}{Definition}
\newtheorem{notation}[equation]{Notation}
\newtheorem*{notation*}{Notation}
\newtheorem{construction}[equation]{Construction}
\newtheorem*{construction*}{Construction}


\theoremstyle{remark}
\newtheorem{remark}[equation]{Remark}
\newtheorem*{remark*}{Remark}
\newtheorem{example}[equation]{Example}
\newtheorem*{example*}{Example}
\newtheorem{warning}[equation]{Warning}
\newtheorem*{warning*}{Warning}
\newtheorem{convention}[equation]{Convention}
\newtheorem*{convention*}{Convention}

\theoremstyle{plain}
\newtheorem{proposition}[equation]{Proposition}
\newtheorem*{proposition*}{Proposition}
\newtheorem{theorem}[equation]{Theorem}
\newtheorem*{theorem*}{Theorem}
\newtheorem{definition-theorem}[equation]{Definition-Theorem}
\newtheorem*{definition-theorem*}{Definition-Theorem}
\newtheorem{postulate}[equation]{Postulate}
\newtheorem*{postulate*}{Postulate}
\newtheorem{lemma}[equation]{Lemma}
\newtheorem*{lemma*}{Lemma}
\newtheorem{observation}[equation]{Observation}
\newtheorem*{observation*}{Observation}
\newtheorem{corollary}[equation]{Corollary}
\newtheorem*{corollary*}{Corollary}
\newtheorem{question}[equation]{Question}
\newtheorem*{question*}{Question}


% <- from preamble


\title{{Good covers and algebras on conically smooth spaces}}
\address{University of Vienna, Faculty of Physics, Mathematical Physics Group, Boltzmanngasse 5, 1090 Vienna, Austria}
\thanks{The authors were supported by the Austrian Science Fund (FWF) through Project no.\ P 37046.}
\author{Aleksandar Ivanov}
\email{aleksandar.ivanov@univie.ac.at}
\author{\"Od\"ul Tet\.{i}k}
\email{oeduel.tetik@univie.ac.at}
\date{}


\begin{document}

\maketitle

\begin{abstract}
    We construct good covers for conically smooth spaces {\color{MidnightBlue} over depth-$1$ posets}. By a result of Karlsson--Scheimbauer--Walde, this implies that, for every such space $X$, constructible factorisation algebras on $X$ and disk algebras over $X$ coincide. We also give a simplified proof of that result.
\end{abstract}



\tableofcontents

\section{Introduction}

\subsubsection*{Acknowledgments} We thank Tashi Walde for very useful exchanges.

\section{Good covers}

\begin{definition}
    Let $X$ be a CSS. A \emph{good cover} of $X$ is an open cover $\mcal{U}$ of $X$ consisting of basics whose finite intersections are also in $\mcal{U}$.
\end{definition}

\begin{definition}
    Let $Y$ be a smooth manifold with or without boundary, equipped with a riemannian metric. We call a cover of $Y$ \emph{geodesically convex} if it consists of geodesically convex basics.
\end{definition}

%It is classical that geodesically convex covers are good. In particular, the union of geodesically convex covers of an arbitrary cover of a smooth manifold with or without boundary is good.

\begin{theorem}
    Let $X$ be a CSS {\color{MidnightBlue} over a depth-$1$ poset}. Then $X$ has a good cover.
\end{theorem}

\begin{proof}
    Without loss of generality, suppose $X$ is stratified over $[1]$ and let $M=X_0$ and $N=X_1$. Recall the blow-up of $M$, the smooth manifold
    \[
        \mrm{Unzip}\cong L\amalg_{L\times(0,\infty)}N
    \]
    with boundary $L=\partial\mrm{Unzip}$. We write $\pi\colon L\to M$ for the accompanying proper fibre bundle. Let us equip $\mrm{Unzip}$ with a riemannian metric which splits along the boundary.

    Using the paracompactness of $M$, let $\mcal{U}$ be a locally finite good cover of $M$ which trivialises $\pi$. For each $U\in\mcal{U}$, let $\epsilon_U>0$ be the radius of injectivity in the normal direction on the compact closure $\overline{U}$ so that there is a smooth embedding 
    \[
        I_U\colon \pi^{-1}U\times[0,\epsilon_U)\hookrightarrow \mrm{Unzip}
    \]
    whose restriction $I_U|_{\pi^{-1}U\times\{0\}}$ to $\pi^{-1}U\subset L$ is given by the boundary inclusion, and the path $I_U(q,-)\colon[0,\epsilon_U)\hookrightarrow \mrm{Unzip}$, for every $q\in\pi^{-1}U$, is the minimising geodesic given by the normal exponential map. More precisely, for $\nu$ the inward-pointing unit normal vector field along the boundary, we set $I_U(x,t)=\exp_x(t\nu_x)$.\footnote{A global $\epsilon$ need not exist unless $\mrm{Unzip}$ is of bounded geometry; see e.g.\ Schick \cite{schick2001bounded}.} Let 
    \[
        C=\bigcup_{U\in\mcal{U}}\mrm{Im}(I_U)
    \]
    be the induced `collar' and consider the cover
    \[  
        \mcal{C}=\{I_U(\pi^{-1}U\times[0,\delta)) : U\in\mcal{U},\ \delta\leq\epsilon_U\}
    \]
    of $C$. Let us write $C_{U,\delta}=I_U(\pi^{-1}U\times[0,\delta))\in\mcal{C}$. Note that $\mcal{C}$ is closed under finite intersections since so is $\mcal{U}$: we have $\pi^{-1}U\cap\pi^{-1}V=\pi^{-1}(U\cap V)$ and
    \[
        C_{U,\delta}\cap C_{V,\delta'}=C_{U\cap V,\min(\delta,\delta')}\in\mcal{C}
    \] 
    since $\min(\delta,\delta')\leq\epsilon_{U\cap V}$ for $\delta\leq\epsilon_U$, $\delta'\leq\epsilon_V$.
    
    Consider now the collection $\mcal{V}$ of those convex geodesic disks $V$ in $N$ such that $V\cap C_{U,\delta}$ is either empty or a convex geodesic disk for every $C_{U,\delta}\in\mcal{C}$. Then $\mcal{V}$ is a good cover of $N$. To prove this, it suffices to show that every $p\in C$ has a neighbourhood $V_p\in\mcal{V}$, since convex geodesic disks are closed under finite intersections and so $V\cap V'\cap C_{U,\delta}=V\cap V''\in\mcal{V}$. Let now $p\in I_U(\pi^{-1}U\times [0,\delta))\in\mcal{C}$, which uniquely determines a point $l_p=\pi(\mrm{pr}_1(p))\in\pi^{-1}U\subset L$. Let $W\subset\pi^{-1}U$ be a convex geodesic disk neighbourhood of $l_p$. Now, since $\mcal{U}$ is locally finite, so is the  cover $\widetilde{\mcal{C}}=\{I_U(\pi^{-1}U\times[0,\epsilon_U))\}$ of $C$ (which is not necessarily closed under finite intersections), so in particular $p$ is contained within finitely many members $I_{U_i}(\pi^{-1}U_i\times[0,\epsilon_i))\in\widetilde{\mcal{C}}$. We necessarily have $p\in I_{U}(\pi^{-1}U\times[0,\min_i(\epsilon_i)))$ with $U\in\{U_i\}_i$ an open satisfying $\epsilon_U=\min_i(\epsilon_i)$. Then $W\times[0,\min_i(\epsilon_i))\ni p$ is a convex geodesic half-disk since the riemannian metric on $\mrm{Unzip}$ splits along the boundary, and so we have $p\in V_p= W\times(0,\min_i(\epsilon_i))\in \mcal{V}$. 

    Let us now observe that $\pi^{-1}U\cong L_p\times U$ for $p\in U$ implies that each
    \[
        C(L_p)\times U\cong U\amalg_{\pi^{-1}U}C_{U,\delta} \subseteq X
    \]
    is a basic in $X$. Thus, writing $\widehat{C_{U,\delta}}=U\amalg_{\pi^{-1}U}C_{U,\delta}$, we obtain that
    \[
        \widehat{\mcal{C}}=\{\widehat{C_{U,\delta}} : C_{U,\delta}\in \mcal{C}\}
    \]
    is a cover by basics of the `tubular neighbourhood' 
    \[
        \widehat{C}=\bigcup_{U\in\mcal{C}}\widehat{C_{U,\epsilon_U}}\subseteq X
    \]
    of $M$. It is closed intersections since so is $\mcal{C}$. Finally, since for $V\in\mcal{V}$ we have $\widehat{C_{U,\delta}}\cap V=C_{U,\delta}\cap V\in\mcal{V}$, we conclude that
    \[
        \widehat{\mcal{C}}\cup \mcal{V}
    \]
    is a good cover of $X$. 
\end{proof}

\section{A proof of...}


\printbibliography

\end{document}