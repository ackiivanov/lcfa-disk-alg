\documentclass[11pt]{amsart}

% from preamble ->

\usepackage[ngerman,french,british]{babel} % clashes with q.uiver/tikzcd
\usepackage[utf8]{inputenc}
\usepackage[T1]{fontenc}
\usepackage{amsthm}
\usepackage{amssymb}
\usepackage{mathtools} % \coloneqq
\usepackage{amsmath}
\usepackage[outline]{contour}
\usepackage[mathscr]{euscript}
\usepackage[all,2cell,cmtip]{xy}
\usepackage{csquotes}
\usepackage[usenames,dvipsnames]{xcolor}
\usepackage[
  %hidelinks
]{hyperref}
\usepackage{microtype}


% quiver ->

% `tikz-cd` is necessary to draw commutative diagrams.
\usepackage{tikz-cd}
% `amssymb` is necessary for `\lrcorner` and `\ulcorner`.
\usepackage{amssymb}
% `calc` is necessary to draw curved arrows.
\usetikzlibrary{calc}
% `pathmorphing` is necessary to draw squiggly arrows.
\usetikzlibrary{decorations.pathmorphing}

\usetikzlibrary{babel} % ADDED MYSELF, OTHERWISE CLASHES WITH TIKZ-CD KEYS -Ö

% A TikZ style for curved arrows of a fixed height, due to AndréC.
\tikzset{curve/.style={settings={#1},to path={(\tikztostart)
    .. controls ($(\tikztostart)!\pv{pos}!(\tikztotarget)!\pv{height}!270:(\tikztotarget)$)
    and ($(\tikztostart)!1-\pv{pos}!(\tikztotarget)!\pv{height}!270:(\tikztotarget)$)
    .. (\tikztotarget)\tikztonodes}},
    settings/.code={\tikzset{quiver/.cd,#1}
        \def\pv##1{\pgfkeysvalueof{/tikz/quiver/##1}}},
    quiver/.cd,pos/.initial=0.35,height/.initial=0}

% TikZ arrowhead/tail styles.
\tikzset{tail reversed/.code={\pgfsetarrowsstart{tikzcd to}}}
\tikzset{2tail/.code={\pgfsetarrowsstart{Implies[reversed]}}}
\tikzset{2tail reversed/.code={\pgfsetarrowsstart{Implies}}}
% TikZ arrow styles.
\tikzset{no body/.style={/tikz/dash pattern=on 0 off 1mm}}

% <- quiver

\usepackage[hyperref,
maxbibnames=99,
backend=biber,
%backref, %& "cit. on p.X",
%backrefstyle=none,
%style=numeric,
style=alphabetic-verb,
%citestyle=numeric,
citestyle=alphabetic-verb,
giveninits=true
%sorting=none
%citestyle=alphabetic
]{biblatex}
\DefineBibliographyStrings{english}{% %REMOVE "p." and "pp."
  page             = {},
  pages            = {},
} 
\renewbibmacro{in:}{} % gets rid of "In:[journal]"
\emergencystretch=1.5em
\addbibresource{text.bib}
\renewcommand*{\bibfont}{\footnotesize}





\usepackage{graphicx}
\usepackage{cleveref}   % use with \cref and \Cref
\usepackage{underscore} % use _ in text



%%%%%%%%%%%INCLUDING INKSCAPE %%%%%%%%%%%%%%
\usepackage{import}
\usepackage{color} % NECESSARY for inkscape for some reason
\usepackage{xifthen}

\usepackage{rotating} 
\newcommand{\incfig}[1]{%
    \def\svgwidth{0.3\columnwidth}
    \import{./figures/}{#1.pdf_tex}
}

\newcommand{\mcal}[1]{\mathcal{#1}}
\newcommand{\msc}[1]{\mathscr{#1}}
\newcommand{\msf}[1]{\mathsf{#1}}
\newcommand{\mscc}[1]{\mathbf{#1}}
\newcommand{\mrm}[1]{\mathrm{#1}}
\newcommand{\mbf}[1]{\mathbf{#1}}
\newcommand{\mbb}[1]{\mathbb{#1}}
\newcommand{\mfrak}[1]{\mathfrak{#1}}
\newcommand{\down}[3][ ]{(#2 \downarrow #3)^{#1}}
\newcommand{\paren}[1]{\left( #1 \right)}


\newcommand{\R}{\mbb{R}}
\DeclareMathOperator{\Imm}{Im}
\newcommand{\id}{\mathrm{id}}
\newcommand{\Hom}{\mathrm{Hom}}
\newcommand{\dfnt}{\bfseries \itshape}
\newcommand{\hyph}{{\mbox{-}}}
\newcommand{\pt}{\mathrm{pt}}
\newcommand{\Fun}{\mathrm{Fun}}
\newcommand{\Tangent}{\mathrm{T}}
\newcommand{\Normal}{\mathrm{N}}
\newcommand{\evv}{\mathrm{ev}}  
\DeclareMathOperator{\Unzip}{Unzip}
\DeclareMathOperator{\Link}{Link}
\DeclareMathOperator{\Span}{span}

\numberwithin{equation}{section}

\theoremstyle{definition}
\newtheorem{definition}[equation]{Definition}%[section]
\newtheorem*{definition*}{Definition}
\newtheorem{notation}[equation]{Notation}
\newtheorem*{notation*}{Notation}
\newtheorem{construction}[equation]{Construction}
\newtheorem*{construction*}{Construction}


\theoremstyle{remark}
\newtheorem{remark}[equation]{Remark}
\newtheorem*{remark*}{Remark}
\newtheorem{example}[equation]{Example}
\newtheorem*{example*}{Example}
\newtheorem{warning}[equation]{Warning}
\newtheorem*{warning*}{Warning}
\newtheorem{convention}[equation]{Convention}
\newtheorem*{convention*}{Convention}

\theoremstyle{plain}
\newtheorem{proposition}[equation]{Proposition}
\newtheorem*{proposition*}{Proposition}
\newtheorem{theorem}[equation]{Theorem}
\newtheorem*{theorem*}{Theorem}
\newtheorem{definition-theorem}[equation]{Definition-Theorem}
\newtheorem*{definition-theorem*}{Definition-Theorem}
\newtheorem{postulate}[equation]{Postulate}
\newtheorem*{postulate*}{Postulate}
\newtheorem{lemma}[equation]{Lemma}
\newtheorem*{lemma*}{Lemma}
\newtheorem{observation}[equation]{Observation}
\newtheorem*{observation*}{Observation}
\newtheorem{corollary}[equation]{Corollary}
\newtheorem*{corollary*}{Corollary}
\newtheorem{question}[equation]{Question}
\newtheorem*{question*}{Question}


% <- from preamble


\title{{Good covers and algebras on conically smooth spaces}}
\address{University of Vienna, Faculty of Physics, Mathematical Physics Group, Boltzmanngasse 5, 1090 Vienna, Austria}
\thanks{The authors were supported by the Austrian Science Fund (FWF) through Project no.\ P 37046.}
\author{Aleksandar Ivanov}
\email{aleksandar.ivanov@univie.ac.at}
\author{\"Od\"ul Tet\.{i}k}
\email{oeduel.tetik@univie.ac.at}
\date{}


\begin{document}

\maketitle

\begin{abstract}
    We construct good covers for conically smooth spaces. By a result of Karlsson--Scheimbauer--Walde, this implies that, for every such space $X$, constructible factorisation algebras on $X$ and disk algebras over $X$ coincide. We also give a simplified proof of that result.
\end{abstract}



\tableofcontents

\section{Introduction}

\subsubsection*{Acknowledgments} We thank Tashi Walde for very useful exchanges.

\section{Localities of the full unzip}


\begin{definition}
    $S_X\subset X$.
\end{definition}

\begin{proposition}\label{7GBVGBA}
    $S_X\subset X$ proper constructible bundle with $\Unzip_{S_X}(X)\cong\Unzip(X)$ and $\Link_{S_X}(X)=\partial\Unzip(X)$.
\end{proposition}


\begin{lemma}\label{DNFREG0}
    {\color{gray}Locally finite good cover on $S_X$. (?)} {\bf unnecessary - second-countability implies more than paracompactness - see [Lee, Smooth, Thm 1.15]}
\end{lemma}

\begin{lemma}\label{K52SWOM}
    Let $X$ be a smooth manifold with corners. Then for every boundary collar $I\colon \R_{\geq0}\times \partial X\hookrightarrow X$ there exists a homeomorphism
    \[
        \alpha_I\colon X^+\xrightarrow{\cong}\R_{\geq0}\times X
    \]
    where $X^+=\bigcup_{r\in\R_{\geq0}}[0,r]\times\partial X\cup_{\{r\}\times\partial X}X$.
\end{lemma}
\begin{proof}
    Let $I\colon\R_{\geq0}\times\partial X\hookrightarrow X$ be a collar, which is a homeomorphism onto its image. We have $X=I(\R_{\geq0}\times \partial X)\cup_{I((0,\infty)\times\partial X)}X^\circ$. Similarly, for each $r$, we have a diffeomorphism 
    \[
        \alpha_r\colon X\to X\smallsetminus I([0,r)\times\partial X)=I([r,\infty)\times\partial X)\cup_{I((r,\infty)\times\partial X)}X\smallsetminus I([0,r]\times\partial X)
    \]
    given by $I(t,q)\mapsto I(t+r,q)$ on $I(\R_{\geq0}\times\partial X)$ and by the identity on $X\smallsetminus \Imm(I)$. We suppress the dependence on $I$ in notation. We obtain a well-defined homeomorphism 
    \[
        I\cup\alpha_r\colon [0,r]\times\partial X\cup_{\{r\}\times \partial X}X\to X,
    \]
    well-defined since $\alpha_r(q)=\alpha_r(I(0,q))=I(r,q)$ for $q\in\partial X$, and thereupon, writing $X^{+r}=[0,r]\times\partial X\cup_{\{r\}\times \partial X}X$, the bijection 
    \begin{align*}
        &\alpha_I\colon X^+\to \R_{\geq0}\times X,\\
        &\alpha|_{X^{+r}}=\{r\}\times I\cup\alpha_r.
    \end{align*}
    We equip $X^+$ with the induced topology, promoting $\alpha$ to a homeomorphism.
\end{proof}


\begin{lemma}\label{DGSR46V}
    Let $X$ be a smooth manifold with corners. Then there is a homeomorphism $\R_{\geq0}\times \partial(\R_{\geq0}\times X)\cong\R_{\geq0} \times X$. Consequently, there is a homeomorphism
    \[
        \Unzip_{C(L)}(C(Z))\cong\R_{\geq0}\times\Link_{C(L)}(C(Z))
    \] 
    for $L=S_Z$.
\end{lemma}

\begin{proof}
    Using \Cref{K52SWOM} it suffices to provide a homeomorphism
    \[
        \phi\colon\R_{\geq0}\times\partial(\R_{\geq0}\times X)\to\bigcup_{r\in\R_{\geq0}}[0,r]\times\partial X\cup_{\{r\}\times\partial X}X.
    \]
    Noting 
    \[
        \partial(\R_{\geq0}\times X)=\{0\}\times X\cup_{\{0\}\times \partial X}\R_{\geq0}\times\partial X,
    \] 
    we define $\phi$ to be the following map: 
    \begin{align*}
        \R_{\geq0}\times\{0\}\times X\ni (t,0,x)&\mapsto x\in X\subset[0,t]\times \partial X\cup_{\{t\}\times\partial X}X\\
        \R_{\geq0}\times\R_{\geq0}\times\partial X\ni(t,s,q)&\mapsto (t,q)\in[0,t+s]\times\partial X.
    \end{align*}
    This map and its inverse $\phi^{-1}$ given by
    \begin{align*}
        [0,r]\times\partial X \ni (t,q) &\mapsto (t,r-t,q)\in \R_{\geq0}\times\R_{\geq0}\times\partial X\\
        [0,r]\times\partial X\cup_{\{r\}\times\partial X}X\supset X\ni x&\mapsto (r,0,x)\in\R_{\geq0}\times\{0\}\times X
    \end{align*}
    are well-defined and continous. Note that $\phi$ does not nepend on a collar.

    The second statement is the special case where $X=\Unzip_L(Z)$ using \Cref{7GBVGBA} and that $\Unzip_{C(L)} C(Z)=\R_{\geq0}\times \Unzip_L(Z)$ and $\Link_{C(L)}C(Z)=\partial\Unzip_{C(L)}(C(Z))$.  
\end{proof}

\begin{remark}
    In the situation of \Cref{K52SWOM} we will regard $X^+$ as a smooth manifold with corners with respect to the smooth structure induced by $\alpha_I$. Up to equivalence, this structure does not depend on the choice of collar. Similarly, in the situation of \Cref{DGSR46V} we will regard $\R_{\geq0}\times\Link_{C(L)}C(Z)$ as a smooth manifold with corners, tautologically diffeomorphic to $\Unzip_{C(L)}C(Z)$ with respect to the induced smooth structure.
\end{remark}


\begin{construction}\label{O6BEXS3}
    Let $X$ be a smooth manifold with corners of dimension $n$, let $\{(U,\phi_U)\}$ a cover of $X$ by coordinate neighbourhoods where each $\phi_U\colon\R^{n-c_U}\times\R^{c_U}_{\geq0}$ is a homeomorphism, and let $\{\rho_U\colon X\to[0,1]\}$ be a partition of unity subordinate to this cover. Recall that a collar $I=I(\{U,\phi_U,\rho_U\})\colon \R_{\geq0}\times\partial X\hookrightarrow X$ is then constructed by defining to be the flow along the nowhere-vanishing inward-pointing vector field $V=\sum\rho_U V_U$ where, in local coordinates, $V_U=\sum_{1\leq i\leq c_U}\partial_i$ where $\{\partial_i\}$ is the standard basis of $\Tangent_0\R^{c_U}_{\geq0}\subset\Tangent_0(\R^{n-c_U}\times\R^{c_U}_{\geq0})$. 

    Let $X$ and $I=I_{\{U,\phi_U,\rho_U\}}$ be as above. Then there is a canonically induced a collar 
    \[
        I^+=I(\{\R_{\geq0}\times U,\id_{\R_{\geq0}}\times\phi_U,\mrm\rho_U\circ\mrm{pr}_X\})\colon \R_{\geq0}\times\partial(\R_{\geq0}\times\partial X)\hookrightarrow \R_{\geq0}\times X
    \] 
    on $\R_{\geq0}\times X$, where $\mrm{pr}_X\colon \R_{\geq}\times X\to X$ is the coordinate projection. It is the flow along the vector field $V^+=\sum(\rho_U\circ\mrm{pr}_X)\cdot V_U^+$ where $V_U^+=\partial_s+V_U$ where $\partial_s$ is the standard basis of $\Tangent_0\R_{\geq0}\subset\Tangent_0(\R_{\geq0}\times\R^{n-c_U}\times\R^{c_U}_{\geq0})$.
\end{construction}

\begin{lemma}\label{DYQ422D}
    Let $X$ be a smooth manifold with corners and let $I^+\colon\R_{\geq0}\times \partial(\R_{\geq0}\times\partial X)\hookrightarrow\R_{\geq0}\times X$ be as in \Cref{O6BEXS3}. Then 
    \[
        I^+=\alpha_I\circ\phi
    \]
    where $\alpha_I$ is as in \Cref{K52SWOM} and $\phi$ is as in the proof of \Cref{DGSR46V}.
\end{lemma}
\begin{proof}
    We observe the restrictions 
    \begin{align*}
        \R_{\geq0}\times\{0\}\times X&\xrightarrow{\alpha_I\circ\phi}\R_{\geq0}\times X\\
        (t,0,x)&\mapsto \begin{cases}
            (t,I(t_x+t,q)), & x=I(t_x,q)\\
            (t,x), & x\in X\smallsetminus\Imm(I)
        \end{cases}
    \end{align*}
    and
    \begin{align*}
        \R_{\geq0}\times\R_{\geq0}\times\partial X&\xrightarrow{\alpha_I\circ \phi} \R_{\geq0} \times X\\
        (t,s,q)&\mapsto (t+s,I(t,q)).
    \end{align*}
    Both maps are the flow (for time $t$) along the vector field $V^+=\sum(\rho_U\circ\mrm{pr}_X)\cdot(\partial_s+V_U)$.
\end{proof}

In the following, $\mbb{D}^n\subset\R^n$ denotes the unit open $n$-disk and $C^{<1}(Z)=\ast\amalg_{\{0\}\times Z}[0,1)\times Z$.

\begin{definition}
    We say an embedded basic $\phi\colon \R^k\times C(Z)\hookrightarrow X$ in a conically smooth space $X$ is \emph{extendable} if there exists an embedding $\widehat{\phi}\colon\R^n\times C(Z)\hookrightarrow X$ such that $\phi$ factors as $\phi\colon\R^{k}\times C(Z)\to \mbb{D}^n\times C^{<1}(Z)\hookrightarrow \R^n\times C^{<1}(Z)\xhookrightarrow{\widehat{\phi}}X$ where the first map is the isomorphism given by the cartesian product of the isomorphisms $\R^n\to\mbb{D}^n$, $x\mapsto \frac{x}{|x|+1}$ and $C(Z)\to C^{<1}(Z)$, $(t,z)\mapsto (\frac{t}{t+1},z)$. 
\end{definition}

\begin{definition}
    Suppose $\mcal{U}$ is a cover of a topological space $X$ which is closed under finite intersections. We say $\mcal{U}$ is \emph{generated} by a cover $\mcal{V}$ and write $\mcal{U}=\langle\mcal{V}\rangle$ if every member of $\mcal{U}$ is a finite intersection of members of $\mcal{V}$.
\end{definition}

\begin{lemma}\label{2978BNV}
    Every smooth manifold $M$ has a good cover $\mcal{U}$ which is generated by extendable basics.
\end{lemma}
\begin{proof}
    Equip $M$ with a riemannian metric. We can put $\mcal{U}=\langle\mcal{V}\rangle$ for $\mcal{V}=\{D_p\}_{p\in M}$ where $D_p$ is the convex disk which is the interior of the image of a ball in $\Tangent_p M$ under the exponential map, with a radius that is strictly smaller than the radius of injectivity.
\end{proof}

\begin{lemma}\label{V6RIX5L}
    Let $L=S_Z$ and let $U=C^{<1}(L)\subset C^{<1}(Z)\subset C(Z)$. Let $\pi\colon \Link(C(Z))|_{U}\to U$ denote the link projection of $\Unzip (C(Z))$ over $U$. Then there is an isomorphism
    \[
        \partial\overline{\pi^{-1}U}\cong \Link_L(Z)
    \]
    and an induced conically smooth collar 
    \(
        (0,1]\times\partial\overline{\pi^{-1}U}\hookrightarrow \overline{\pi^{-1}U}
    \)
    which is a refinement onto its image.
\end{lemma}
\begin{proof}
    Using \Cref{7GBVGBA} we have $\Unzip C(Z)=\Unzip_{S}C(Z)=\R_{\geq0}\times\Unzip_L Z$ and so 
    \begin{align*}
        \Unzip C(Z)|_{U}&=\Link_{C(L)}C(Z)|_{C^{<1}(L)}\\
        &=\{0\}\times \Unzip_L Z\cup_{\{0\}\times\Link_LZ}[0,1)\times\Link_L Z\\
    \end{align*}
    since the projection $\pi\colon\Link_{C(L)}C(Z)=\{0\}\times\Unzip_L Z\cup_{\{0\}\times\Link_LZ}\R_{\geq0}\times \Link_L Z\to C(L)=\ast\amalg_{\{0\}\times L}\R_{\geq0}\times L$ is given by mapping all of $\{0\}\times\Unzip_L Z$ to $\ast$ and on $\R_{\geq0}\times \Link_LZ$ by $\id_{\R_{\geq0}}\times\pi'$ where $\pi'\colon \Link_LZ\to L$ is the link projection.
    Thus 
    \[
        \pi^{-1}\overline{U}=\pi^{-1}C^{\leq 1}(U)=\{0\}\times\Unzip_LZ\cup[0,1]\times\Link_LZ 
    \]
    and consequently 
    \begin{equation}\label{QX54WZB}
        \pi^{-1}\partial\overline{U}=\pi^{-1}\{1\}\times L=\{1\}\times\Link_L Z,
    \end{equation}
    proving the first claim. The collar is immediate.
\end{proof}

\begin{example}
    
\end{example}

\begin{lemma}\label{FBLD2TE}
    Let $\R^n$ be equipped with a riemannian metric, and let $c\in\{0,\dots,n\}$. Then for all geodesic disks $D\subset\R^n$ about the origin the intersections
    \begin{enumerate}
        \item $D\cap (\R^{n-c}\times(\R^{c}\smallsetminus\R^{c}_{\geq0}))$
        \item $D \cap (\R^{n-c}\times\R^{c}_{>0})$
    \end{enumerate}
    are disks.
\end{lemma}
\begin{proof}
    Let us suppose $n=c$ for simplicity. For $r>0$, recall the diffeomorphism $[0,r)\to[0,\infty)$, $x\mapsto\frac{x}{r-x}$. In the same way, $\rho_r\colon B_r\to\R^n$, $v\mapsto \frac{d(0,v)}{r-d(0,v)}v$ is a diffeomorphism from the disk $B_r=\{x:\R^n : d(0,x)<r\}$ of radius $r$ about the origin to all of $\R^n$, where $d$ is the metric associated with the given riemannian metric. Now we need only note that $\rho_r$ restricts to diffeomorphisms $B_r\cap\R^n_{>0}\to\R^n_{>0}$ and $B_r\cap(\R^n\smallsetminus\R^n_{\geq0})\to\R^n\smallsetminus\R^n_{\geq0}$, and that both targets are disks.
\end{proof}


\begin{proposition}
    Let $L=S_Z$ and let $I=J^+\colon \R_{\geq0}\times\Link_{C(L)}C(Z)\hookrightarrow\Unzip_{C(L)}C(Z)$ be the collar induced, according to \Cref{O6BEXS3}, by a collar $J\colon\R_{\geq0}\times\Link_L(Z)\hookrightarrow \Unzip_L(Z)$. Let $U$ be as in \Cref{V6RIX5L}, and let 
    \[
        p\in I(\{1\}\times\partial\overline{\pi^{-1}U}).
    \]
    Then:
    \begin{enumerate} 
        \item There is a diffeomorphism 
        \[
            \R_{>1}\times J(\R_{>0}\times\Link_LZ)\cap I([0,1]\times\overline{\pi^{-1}U})\cong (1,2]\times J((0,1]\times\Link_LZ)
        \]
        of smooth manifolds with corners.
        \item There exists a convex disk $D\subset\Unzip_{C(L)}C(Z)^\circ$ about $p$ such that $D\cap I([0,1)\times\pi^{-1}U)$ is a disk. 
        % \item There exists a chart neighbourhood $V\subset\Unzip_{C(L)}C(Z)^\circ$ of $p$ which is diffeomorphic to $\R_{>0}\times J(\R_{>0}\times W)$ where $W\subset\Link_L(Z)$ is the boundary of a chart neighbourhood in $\Unzip_LZ$.
        % \item There exists a homeomorphism 
        % \[
        %     \Psi\colon\R^2\times\Link_LZ\to \R_{>1}\times J(\R_{>0}\times\Link_LZ)
        % \]
        % such that 
        % \[
        %     \R_{>1}\times J(\R_{>0}\times\Link_LZ)\cap I([0,1]\times\overline{\pi^{-1}U})=\Psi(\R_{\leq0}^2\times\Link_LZ).
        % \]
    \end{enumerate}
\end{proposition}
\begin{proof}
    From the proof of \Cref{V6RIX5L} we recall that
    \(
        \overline{\pi^{-1}U}=\{0\}\times\Unzip_LZ\cup_{\{0\}\times\Link_LZ}[0,1]\times\Link_LZ.
    \)
    By \Cref{DYQ422D} we have $I=\alpha_J\circ\phi$, and observe that 
    \begin{align*}
        \phi([0,1]\times\overline{\pi^{-1}U})&= A'\cup B'\\
        &\coloneqq\bigcup_{t\in[0,1]}(X\subset X^{+t})\cup\{(t,q)\in X^{+(t+s)} : t,s\in[0,1]\}.
    \end{align*}
    Thus $I([0,1]\times\overline{\pi^{-1}U})=A\cup B\subset\Unzip_{C(L)}C(Z)=\R_{\geq0}\times\Unzip_LZ$ where $A=\alpha_J(A')$, $B=\alpha_J(B')$. We have 
    \begin{align*}
        &A=\{(r,x)\in\R_{\geq0}\times\Unzip_LZ : r\in[0,1],\ x\in \Unzip_LZ\smallsetminus J([0,r)\times\Link_LZ\}\\
        &B=\{(t+s,J(t,q))\in\R_{\geq0}\times\Unzip_LZ : t,s\in[0,1]\}
    \end{align*}
    and hence
    \[
        \R_{>1}\times J(\R_{>0}\times\Link_LZ)\cap(A\cup B)=\{(t+s,J(t,q)) : t\in(0,1],\ s\in[0,1],\ t+s>1\}
    \]
    as the intersection with $A$ is empty.

    Consider now the diffeomorphism 
    \begin{align*}
    \psi=\rho\circ\begin{pmatrix}
        1 & 1\\ 
        0 & 1
    \end{pmatrix}\circ\rho^{-1}\colon\R_{>1}\times\R_{>0}&\to\R_{>1}\times\R_{>0},\\
     (x,y)&\mapsto ((x-1)y+1,y)
    \end{align*} 
    where $\rho\colon\R^2\to\R_{>1}\times\R_{>0}$, $(x,y)\mapsto(e^x+1,e^y)$. Now $\psi$ fixes $(2,1)$ and satisfies 
    \[
        \psi((1,2]\times(0,1])=\{(t+s,t) : t\in(0,1],\ s\in[0,1],\ t+s>1\}.
    \] 
    Putting 
    \[
        \Psi=\id_{\R_{>1}}\times J\circ\psi\times\id_{\Link_LZ}\colon\R_{>1}\times\R_{>0}\times\Link_LZ\to\R_{>1}\times J(\R_{>0}\times\Link_LZ)
    \] 
    we obtain the map
    \begin{align*}
        (1,2]\times J((0,1]\times\Link_LZ)&\xrightarrow{\cong}(1,2]\times(0,1]\times\Link_LZ\\
        &\xrightarrow{\Psi|}\R_{>1}\times J(\R_{>0}\times\Link_LZ)\cap I([0,1]\times\overline{\pi^{-1}U}).
    \end{align*}
    The first map, which is a homeomorphism, is tautologically a diffeomorphism of smooth manifolds with corners with respect to the induced smooth structure on its target. With respect to the latter, $\Psi$ is a diffeomorphism as well. Hence the composition is a diffeomorphism. In particular, we obtain the restricted diffeomorphism
    \[
        (1,2)\times J((0,1)\times\Link_LZ)\cong\R_{>1}\times J(\R_{>0}\times\Link_LZ)\cap C_{U}.
    \]
    of smooth manifolds without boundary.

    Finally, consider $p=I(1,(1,q))=(2,J(1,q))$ where $q\in\Link_LZ$ (recall \eqref{QX54WZB}). Let $\R^{n-c}\times\R^{c}_{\geq0}\cong W\subset\Unzip_LZ$ be a chart neighbourhood of $q$. Without loss of generality we may assume $c=n$ for simplicity. Up to diffeomorphism we may write the restriction of the collar of $\Link_LZ$ as $J\colon\R_{\geq0}\times\partial\R^c_{\geq0})\hookrightarrow \R^c_{\geq0}$, given by the flow along the vector field $\sum_{1\leq i\leq c}\partial_i$ where $\{\partial_i\}$ is the standard basis of $\Tangent_0\R^c_{\geq0}$ (recall \Cref{O6BEXS3}). We now observe the diffeomorphism 
    \[
        J((0,1]\times\partial \R^{c}_{\geq0})=\R^c_{>0}\smallsetminus\left(\begin{pmatrix}
        1 \\ \vdots \\ 1
        \end{pmatrix}
        +\R^c_{\geq0}\right)
        \cong\R^c\smallsetminus\R^c_{>0}
    \]
    of smooth manifolds with corners.
    Consequently we have $(1,2]\times J((0,1]\times\partial\R^c_{\geq0})\cong\R^{c+1}\smallsetminus\R^{c+1}_{>0}$. These diffeomorphisms restrict to $J((0,1)\times \partial\R^c_{\geq0})\cong\R^c\smallsetminus\R^c_{\geq0}$ and so $(1,2)\times J((0,1)\times\partial\R^c_{\geq0})\cong \R^{c+1}\smallsetminus\R^{c+1}_{\geq0}$. Hence, the intersection of any convex disk $D\subset\R_{>1}\times J(\R_{>0}\times W)$ centred at $(2,J(1,q))=p$ with $I([0,1)\times\pi^{-1}U)$ is a disk by the first statement of \Cref{FBLD2TE}.
\end{proof}

\begin{example}
    
\end{example}

\begin{corollary}
    
\end{corollary}


\section{Good covers}

\begin{definition}
    Let $X$ be a CSS. A \emph{good cover} of $X$ is an open cover $\mcal{U}$ of $X$ consisting of basics whose finite intersections are also in $\mcal{U}$.
\end{definition}

\begin{definition}
    Let $Y$ be a smooth manifold with or without boundary, equipped with a riemannian metric. We call a cover of $Y$ \emph{geodesically convex} if it consists of geodesically convex basics.
\end{definition}

%It is classical that geodesically convex covers are good. In particular, the union of geodesically convex covers of an arbitrary cover of a smooth manifold with or without boundary is good.

\begin{theorem}
    Let $X$ be a CSS {\color{MidnightBlue} over a depth-$1$ poset}. Then $X$ has a good cover.
\end{theorem}

\begin{proof}
    Without loss of generality, suppose $X$ is stratified over $[1]$ and let $M=X_0$ and $N=X_1$. Recall the blow-up of $M$, the smooth manifold
    \[
        \mrm{Unzip}\cong L\amalg_{L\times(0,\infty)}N
    \]
    with boundary $L=\partial\mrm{Unzip}$. We write $\pi\colon L\to M$ for the accompanying proper fibre bundle. Let us equip $\mrm{Unzip}$ with a riemannian metric which splits along the boundary.

    Using the paracompactness of $M$, let $\mcal{U}$ be a locally finite good cover of $M$ which trivialises $\pi$. For each $U\in\mcal{U}$, let $\epsilon_U>0$ be the radius of injectivity in the normal direction on the compact closure $\overline{U}$ so that there is a smooth embedding 
    \[
        I_U\colon \pi^{-1}U\times[0,\epsilon_U)\hookrightarrow \mrm{Unzip}
    \]
    whose restriction $I_U|_{\pi^{-1}U\times\{0\}}$ to $\pi^{-1}U\subset L$ is given by the boundary inclusion, and the path $I_U(q,-)\colon[0,\epsilon_U)\hookrightarrow \mrm{Unzip}$, for every $q\in\pi^{-1}U$, is the minimising geodesic given by the normal exponential map. More precisely, for $\nu$ the inward-pointing unit normal vector field along the boundary, we set $I_U(x,t)=\exp_x(t\nu_x)$.\footnote{A global $\epsilon$ need not exist unless $\mrm{Unzip}$ is of bounded geometry; see e.g.\ Schick \cite{schick2001bounded}.} Let 
    \[
        C=\bigcup_{U\in\mcal{U}}\mrm{Im}(I_U)
    \]
    be the induced `collar' and consider the cover
    \[  
        \mcal{C}=\{I_U(\pi^{-1}U\times[0,\delta)) : U\in\mcal{U},\ \delta\leq\epsilon_U\}
    \]
    of $C$. Let us write $C_{U,\delta}=I_U(\pi^{-1}U\times[0,\delta))\in\mcal{C}$. Note that $\mcal{C}$ is closed under finite intersections since so is $\mcal{U}$: we have $\pi^{-1}U\cap\pi^{-1}V=\pi^{-1}(U\cap V)$ and
    \[
        C_{U,\delta}\cap C_{V,\delta'}=C_{U\cap V,\min(\delta,\delta')}\in\mcal{C}
    \] 
    since $\min(\delta,\delta')\leq\epsilon_{U\cap V}$ for $\delta\leq\epsilon_U$, $\delta'\leq\epsilon_V$.
    
    Consider now the collection $\mcal{V}$ of those convex geodesic disks $V$ in $N$ such that $V\cap C_{U,\delta}$ is either empty or a convex geodesic disk for every $C_{U,\delta}\in\mcal{C}$. Then $\mcal{V}$ is a good cover of $N$. To prove this, it suffices to show that every $p\in C$ has a neighbourhood $V_p\in\mcal{V}$, since convex geodesic disks are closed under finite intersections and so $V\cap V'\cap C_{U,\delta}=V\cap V''\in\mcal{V}$. Let now $p\in I_U(\pi^{-1}U\times [0,\delta))\in\mcal{C}$, which uniquely determines a point $l_p=\pi(\mrm{pr}_1(p))\in\pi^{-1}U\subset L$. Let $W\subset\pi^{-1}U$ be a convex geodesic disk neighbourhood of $l_p$. Now, since $\mcal{U}$ is locally finite, so is the  cover $\widetilde{\mcal{C}}=\{I_U(\pi^{-1}U\times[0,\epsilon_U))\}$ of $C$ (which is not necessarily closed under finite intersections), so in particular $p$ is contained within finitely many members $I_{U_i}(\pi^{-1}U_i\times[0,\epsilon_i))\in\widetilde{\mcal{C}}$. We necessarily have $p\in I_{U}(\pi^{-1}U\times[0,\min_i(\epsilon_i)))$ with $U\in\{U_i\}_i$ an open satisfying $\epsilon_U=\min_i(\epsilon_i)$. Then $W\times[0,\min_i(\epsilon_i))\ni p$ is a convex geodesic half-disk since the riemannian metric on $\mrm{Unzip}$ splits along the boundary, and so we have $p\in V_p= W\times(0,\min_i(\epsilon_i))\in \mcal{V}$. 

    Let us now observe that $\pi^{-1}U\cong L_p\times U$ for $p\in U$ implies that each
    \[
        C(L_p)\times U\cong U\amalg_{\pi^{-1}U}C_{U,\delta} \subseteq X
    \]
    is a basic in $X$. Thus, writing $\widehat{C_{U,\delta}}=U\amalg_{\pi^{-1}U}C_{U,\delta}$, we obtain that
    \[
        \widehat{\mcal{C}}=\{\widehat{C_{U,\delta}} : C_{U,\delta}\in \mcal{C}\}
    \]
    is a cover by basics of the `tubular neighbourhood' 
    \[
        \widehat{C}=\bigcup_{U\in\mcal{C}}\widehat{C_{U,\epsilon_U}}\subseteq X
    \]
    of $M$. It is closed intersections since so is $\mcal{C}$. Finally, since for $V\in\mcal{V}$ we have $\widehat{C_{U,\delta}}\cap V=C_{U,\delta}\cap V\in\mcal{V}$, we conclude that
    \[
        \widehat{\mcal{C}}\cup \mcal{V}
    \]
    is a good cover of $X$. 
\end{proof}

\section{A proof of...}


\printbibliography

\end{document}